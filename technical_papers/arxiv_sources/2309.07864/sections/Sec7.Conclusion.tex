\section{Conclusion}


This paper provides a comprehensive and systematic overview of LLM-based agents, discussing the potential challenges and opportunities in this flourishing field. 
We begin with a philosophical perspective, elucidating the origin and definition of agent, it evolution in the field of AI, and why LLMs are suited to serve as the main part of the brain of agents. 
Motivated by these background information, we present a general conceptual framework for LLM-based agents, comprising three main components: the brain, perception, and action. Next, we introduce the wide-ranging applications of LLM-based agents, including single-agent applications, multi-agent systems, and human-agent collaboration. Furthermore, we move beyond the notion of agents merely as assistants, exploring their social behavior and psychological activities, and situating them within simulated social environments to observe emerging social phenomena and insights for humanity.
Finally, we engage in discussions and offer a glimpse into the future, touching upon the mutual inspiration between LLM research and agent research, the evaluation of LLM-based agents, the risks associated with them, the opportunities in scaling the number of agents, and some open problems like Agent as a Service and whether LLM-based agents represent a potential path to AGI.
We hope our efforts can provide inspirations to the community and facilitate research in related fields.
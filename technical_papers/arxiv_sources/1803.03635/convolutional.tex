\section{Winning Tickets in Convolutional Networks}
\label{sec:conv}

Here, we apply the lottery ticket hypothesis to convolutional networks on CIFAR10,
increasing both the complexity of the learning problem and the size of the networks.
We consider the Conv-2, Conv-4, and Conv-6 architectures in Figure \ref{fig:convnets}, which are scaled-down variants
of the VGG \citep{vgg} family. The networks have two, four, or six convolutional layers followed by
two fully-connected layers; max-pooling occurs after every two convolutional layers. The networks cover a range from near-fully-connected to
traditional convolutional networks, with less than 1\% of parameters
in convolutional layers in Conv-2 to nearly two thirds in Conv-6.%
\footnote{Appendix \ref{app:conv} explores other hyperparameters, including learning rates, optimization strategies (SGD, momentum),
and the relative rates at which to prune convolutional and fully-connected layers.}
%In this Section, we investigate how these small convolutional networks behave under the lottery ticket experiment.

\begin{figure}
\centering
\includegraphics[width=.8\textwidth]{graphs/cifar10/conv/conv_paperall_w/legend}%
\vspace{-1em}
\includegraphics[width=.5\textwidth]{graphs/cifar10/conv/conv_paperall_w/iteration}%
\includegraphics[width=.5\textwidth]{graphs/cifar10/conv/conv_paperall_w/accuracy}%
\vspace{-1em}
\includegraphics[width=.5\textwidth]{graphs/cifar10/conv/conv_paperall_w/train_accuracy}%
\includegraphics[width=.5\textwidth]{graphs/cifar10/conv/conv_paperall_w_last_iteration/accuracy}%
\vspace{-.5em}
\caption{Early-stopping iteration and test and training accuracy of the Conv-2/4/6 architectures when iteratively pruned and when
randomly reinitialized.
Each solid line is the average of five trials; each dashed line is the average of fifteen reinitializations
(three per trial). The bottom right graph plots test accuracy of winning tickets at iterations corresponding to the last iteration of training for the original network (20,000 for Conv-2, 25,000 for Conv-4, and 30,000 for Conv-6); at this iteration, training accuracy $\approx 100\%$ for $P_m \geq 2\%$ for winning tickets (see Appendix \ref{app:train}).}
\label{fig:question-1}
\end{figure}

\textbf{Finding winning tickets.} The solid lines in Figure \ref{fig:question-1} (top) show the iterative lottery ticket experiment on
Conv-2 (blue), Conv-4 (orange), and Conv-6 (green) at the per-layer pruning rates from Figure \ref{fig:convnets}. The pattern
from Lenet in Section \ref{sec:fc} repeats: as the network is pruned, it learns faster and test accuracy rises as compared to the original network.
In this case, the results are more pronounced.
Winning tickets reach minimum validation loss at best 3.5x faster for Conv-2 ($P_m = 8.8\%$), 3.5x for Conv-4 ($P_m = 9.2\%$), and
2.5x for Conv-6 ($P_m = 15.1\%$). Test accuracy improves at best 3.4 percentage points for Conv-2 ($P_m = 4.6\%$), 3.5
for Conv-4 ($P_m = 11.1\%$), and 3.3 for Conv-6 ($P_m = 26.4\%$). All three networks remain above their original average test accuracy when $P_m > 2\%$.

As in Section \ref{sec:fc}, training accuracy at the early-stopping iteration rises with test accuracy.
However,  at iteration 20,000 for Conv-2, 25,000 for Conv-4, and 30,000 for Conv-6 (the iterations corresponding to the final training iteration for the original network), training accuracy reaches 100\% for all networks when $P_m \geq 2\%$ (Appendix \ref{app:train}, Figure \ref{fig:question-12})
and winning tickets still maintain higher test accuracy (Figure \ref{fig:question-1} bottom right). This means that the gap between
test and training accuracy is smaller for winning tickets, indicating they generalize better.

\textbf{Random reinitialization.} We repeat the random reinitialization experiment from Section \ref{sec:fc}, which appears as the dashed lines in Figure \ref{fig:question-1}. 
These networks again take increasingly longer to learn upon continued pruning. Just as with Lenet on MNIST (Section \ref{sec:fc}), test accuracy drops off more
quickly for the random reinitialization experiments. However, unlike Lenet, test accuracy at early-stopping time initially remains steady and even
improves for Conv-2 and Conv-4, indicating that---at moderate levels of pruning---the structure of the winning tickets alone may lead to better accuracy.
\section{Discussion}

Existing work on neural network pruning (e.g., \citet{han-pruning}) demonstrates that the function
learned by a neural network can often be represented with fewer parameters. Pruning typically proceeds by training the original network, removing connections, and further fine-tuning. 
In effect, the initial training initializes the weights of the
pruned network so that it can learn in isolation during fine-tuning.  
We seek to determine if similarly sparse networks can learn from the
start.
We find that the architectures studied in this paper reliably contain such trainable subnetworks, and the lottery ticket hypothesis proposes that this property applies in general.
Our empirical study of the existence and nature of winning tickets invites a number of follow-up questions.


\textbf{The importance of winning ticket initialization.}
When randomly reinitialized, a winning ticket learns more slowly and achieves lower test accuracy, suggesting
that initialization is important to its success. One
possible explanation for this behavior is these initial weights are close to their
final values after training---that in the most extreme case, they are already trained. However,
experiments in Appendix \ref{app:examining} show the opposite---that
the winning ticket weights move further
 than
other weights.  This suggests that the benefit of the initialization is connected
to the optimization algorithm, dataset, and model. For example,
the winning ticket initialization might land in a region of the loss landscape
that is particularly amenable to optimization by the chosen optimization algorithm.
%TODO\todo{cite landscape papers here, including entropy-sgd, smooth minima (hinton), landscape visualization}

\citet{rethinking-pruning} find that pruned networks are indeed trainable when randomly reinitialized,
seemingly contradicting conventional wisdom and our random reinitialization experiments.
For example, on VGG-19 (for which we share the same setup),
they find that networks pruned by up to 80\% and randomly reinitialized match the accuracy of the original network.
Our experiments in Figure \ref{fig:vgg} confirm these findings at this level of sparsity  (below which \citeauthor{rethinking-pruning} do not present data). However, after further pruning, initialization matters:
we find winning tickets when VGG-19 is pruned by up to 98.5\%; when reinitialized, these tickets reach much lower accuracy.
We hypothesize that---up to a certain level of sparsity---highly overparameterized networks can be pruned, reinitialized, and retrained successfully; however, beyond this point, extremely pruned, less severely overparamterized networks only maintain
accuracy with fortuitous initialization.

%\NA{\citet{xavier} and \citet{he} propose random initialization schemes for dense neural networks designed to ``maintain magnitudes...of
%gradients.'' However, we are not aware of any
%work studying good initializations for sparse networks. Furthermore, although we initialize our unpruned networks using the
%Glorot scheme, we do not rescale the weights of winning tickets after pruning (as do \citet{rethinking-pruning}) to maintain
%the gradient propagation properties of the original initialization.}

\textbf{The importance of winning ticket structure.}
The initialization that gives rise to a winning ticket is arranged in a particular sparse architecture.
Since we uncover winning tickets through heavy use of training data, we hypothesize that the structure of
our winning tickets encodes an inductive bias customized to the learning task at hand.  \citet{inductive-bias} show that
the inductive bias embedded in the structure of a deep network determines the kinds of data that it can
separate more parameter-efficiently than can a shallow network; although \citet{inductive-bias} focus on the
pooling geometry of convolutional networks, a similar effect may be at play with the structure of winning tickets, allowing them
to learn even when heavily pruned.

\textbf{The improved generalization of winning tickets.}
We reliably find winning tickets that generalize better, exceeding the test accuracy of the original
network while matching its training accuracy.
Test accuracy increases and then decreases as we prune, forming an {\em Occam's Hill}
~\citep{occam} where the original, overparameterized model has too much
complexity (perhaps overfitting) and the
extremely pruned model has too little.
%
The conventional view of the relationship between compression and generalization is that compact hypotheses
can better generalize \citep{mdl}. Recent theoretical work shows a similar link for neural networks,
proving tighter generalization bounds for networks that can be compressed further
(\citet{comp} for pruning/quantization and \citet{arora} for noise robustness). The lottery ticket hypothesis offers a
complementary perspective on this relationship---that larger networks might explicitly contain simpler representations.
% TODO: Flesh this point out for camera ready.
%In addition, these bounds do not yet consider initialization or optimization algorithms, which
%impact the generalization of the compressed networks that we find.

\textbf{Implications for neural network optimization.}
%\footnote{Not all dense neural networks can contain winning tickets. For example, a minimal relu network with two hidden
%units for the two-way xor function is dense, but removing any one parameter makes it impossible to match the accuracy of the
%original network.}
Winning tickets can reach accuracy equivalent to that of the
original, unpruned network, but with significantly fewer parameters.
This observation connects to recent work on the role of
overparameterization in neural network training.
For example,
\NA{\citet{provably} prove that sufficiently overparameterized two-layer relu
networks (with fixed-size second layers) trained with SGD converge to
global optima.} A key question, then,  is whether the presence of a winning ticket is
necessary or sufficient for SGD to optimize a neural network to a
particular test accuracy. We conjecture (but do not empirically show) that
SGD seeks out and trains a well-initialized subnetwork. By this
logic, overparameterized networks are easier to train because they have more
combinations of subnetworks that are potential
winning tickets.
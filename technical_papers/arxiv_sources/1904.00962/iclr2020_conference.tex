
\documentclass{article} % For LaTeX2e
\usepackage{iclr2020_conference,times}

% Optional math commands from https://github.com/goodfeli/dlbook_notation.
\input{math_commands.tex}

\usepackage{url}

\usepackage{graphicx} % more modern
\usepackage{subfigure} 
\usepackage{multirow}
\usepackage{xcolor}
\usepackage{mathtools}
\usepackage{relsize}
\usepackage{nicefrac}
\usepackage{xspace}
\usepackage{amsmath,amssymb,enumerate}
\usepackage{amsthm,cancel}
\usepackage{natbib}
\usepackage{enumitem}
\usepackage{dsfont}
\usepackage[colorlinks=true,citecolor=blue]{hyperref}
\usepackage[utf8]{inputenc} % allow utf-8 input
\usepackage[T1]{fontenc}    % use 8-bit T1 fonts

\newtheorem{lemma}{Lemma}
\newtheorem{theorem}{Theorem}
\newtheorem{corollary}{Corollary}
\newtheorem{remark}{Remark}
\newtheorem{definition}{Definition}
\newtheorem{assumption}{Assumption}

\newtheorem*{lemma*}{Lemma}
\newtheorem*{theorem*}{Theorem}
\newtheorem*{assumption*}{Assumption}
\newtheorem*{corollary*}{Corollary}
\newtheorem*{remark*}{Remark}
\newtheorem*{definition*}{Definition}

\newcommand{\resnet}{\textsc{ResNet}}
\newcommand{\bert}{\textsc{Bert}\xspace}


\newcommand{\sag}{\textsc{Sag}}
\newcommand{\sgd}{\textsc{Sgd}\xspace}
\newcommand{\sdca}{\textsc{Sdca}}
\newcommand{\csgd}{\textsc{CSgd}}
\newcommand{\dsgd}{\textsc{DSgd}}
\newcommand{\saga}{\textsc{Saga}}
\newcommand{\stgd}{\textsc{S2gd}}
\newcommand{\svrg}{\textsc{Svrg}\xspace}
\newcommand{\msvrg}{\textsc{Msvrg}}
\newcommand{\gd}{\textsc{GradientDescent}\xspace}
\newcommand{\adagrad}{\textsc{Adagrad}}
\newcommand{\nadam}{\textsc{Nadam}}
\newcommand{\adadelta}{\textsc{Adadelta}}
\newcommand{\adam}{\textsc{Adam}}
\newcommand{\adamnc}{\textsc{AdamNc}}
\newcommand{\adamw}{\textsc{AdamW}}
\newcommand{\rmsprop}{\textsc{RMSprop}}
\newcommand{\amsgrad}{\textsc{AMSGrad}}
\newcommand{\yogi}{\textsc{Yogi}}
\newcommand{\lamb}{\textsc{Lamb}}
\newcommand{\lars}{\textsc{Lars}}
\newcommand{\avsgrad}{\textsc{AVSGrad}}
\newcommand{\cifarnet}{\textsc{Cifarnet}}
\newcommand{\diag}{\text{diag}\xspace}
\newcommand{\sml}[1]{{\small #1}}
\newcommand{\fromto}[3]{\sml{$#1{\;\le\;}#2{\;\le\;}#3$}}

\newcommand{\reals}{\mathbb{R}}
\newcommand{\note}[1]{\textcolor{red}{#1}}
\newcommand{\ip}[2]{\left\langle #1, #2\right\rangle}
\newcommand{\nlsum}{\sum\nolimits}
%\newcommand{\E}{\mathbb{E}}
\newcommand{\Fc}{\mathcal{F}}

\renewcommand{\baselinestretch}{0.97}
\frenchspacing
\sloppy

\usepackage{hyperref}
\usepackage{url}

\usepackage[utf8]{inputenc} % allow utf-8 input
\usepackage[T1]{fontenc}    % use 8-bit T1 fonts
\usepackage{url}            % simple URL typesetting
\usepackage{booktabs}       % professional-quality tables
\usepackage{amsfonts}       % blackboard math symbols
\usepackage{nicefrac}       % compact symbols for 1/2, etc.
\usepackage{microtype}      % microtypography
%\usepackage{devanagari}
\usepackage{algorithm}
\usepackage{algorithmic}

\title{Large Batch Optimization for Deep Learning: Training BERT in 76 minutes}

% Authors must not appear in the submitted version. They should be hidden
% as long as the \iclrfinalcopy macro remains commented out below.
% Non-anonymous submissions will be rejected without review.

\author{\small Yang You$^{2}$, 
%\footnote{Work was performed when Y.You was a student researcher at Google Brain}
Jing Li$^{1}$, Sashank Reddi$^{1}$, Jonathan Hseu$^{1}$, Sanjiv Kumar$^{1}$, Srinadh Bhojanapalli$^{1}$\\{\bf \small Xiaodan Song}$^{1}$, {\bf \small James Demmel}$^{2}$, {\bf \small Kurt Keutzer}$^{2}$, {\bf \small Cho-Jui Hsieh}$^{1,3}$\\
{\scriptsize Yang You was a student researcher at Google Brain. This project was done when he was at Google Brain.}
\\
\normalsize{Google$^{1}$},
\normalsize{UC Berkeley$^{2}$},
\normalsize{UCLA$^{3}$}\\
\scriptsize{\{youyang, demmel, keutzer\}@cs.berkeley.edu, \{jingli, sashank, jhseu, sanjivk, bsrinadh, xiaodansong, chojui\}@google.com}
}

% The \author macro works with any number of authors. There are two commands
% used to separate the names and addresses of multiple authors: \And and \AND.
%
% Using \And between authors leaves it to \LaTeX{} to determine where to break
% the lines. Using \AND forces a linebreak at that point. So, if \LaTeX{}
% puts 3 of 4 authors names on the first line, and the last on the second
% line, try using \AND instead of \And before the third author name.

\newcommand{\fix}{\marginpar{FIX}}
\newcommand{\new}{\marginpar{NEW}}

\iclrfinalcopy % Uncomment for camera-ready version, but NOT for submission.
\begin{document}


\maketitle

\begin{abstract}
Training large deep neural networks on massive datasets is  computationally very challenging. There has been recent surge in interest in using \emph{large batch} stochastic optimization methods to tackle this issue. The most prominent algorithm in this line of research is $\lars$, which by  employing \emph{layerwise adaptive} learning rates trains $\resnet$ on ImageNet in a few minutes. However, $\lars$ performs poorly for attention models like $\bert$, indicating that its performance gains are \emph{not} consistent across tasks. In this paper, we first study a principled layerwise adaptation strategy to accelerate training of deep neural networks using large mini-batches. Using this strategy, we develop a new layerwise adaptive large batch optimization technique called $\lamb$; we then provide convergence analysis of $\lamb$ as well as $\lars$, showing convergence to a stationary point in general nonconvex settings. Our empirical results demonstrate the superior performance of $\lamb$ across various tasks such as $\bert$ and $\resnet$-50 training with very little hyperparameter tuning. In particular, for $\bert$ training, our optimizer enables use of very large batch sizes of 32868 without any degradation of performance.  By increasing the batch size to the memory limit of a TPUv3 Pod, $\bert$ training time can be reduced from 3 days to just 76 minutes (Table \ref{table:results}). The $\lamb$ implementation is available online\footnote{\url{https://github.com/tensorflow/addons/blob/master/tensorflow_addons/optimizers/lamb.py}}.
\end{abstract}

\iffalse
Training large deep neural networks on massive datasets is  computationally very challenging. There has been recent surge in interest in using \emph{large batch} stochastic optimization methods to tackle this issue. The use of large batch of gradients (which are computed in parallel) in these methods enable them to use much larger learning rates, thereby, drastically reducing the training time. However, current methods require extensive tuning of learning rates and their performance gains are \emph{not} consistent across tasks. In this paper, we study a principled adaptation strategy to accelerate training of deep neural networks using large mini-batches. Using this strategy, we develop a new layer-wise adaptive large batch optimization technique called $\lamb$. We provide a formal convergence analysis of $\lamb$ as well as the previous published layerwise optimizer $\lars$, showing convergence to a stationary point in general nonconvex settings. Our empirical results demonstrate the superior performance of $\lamb$ across various tasks such as BERT and ResNet-50 training with very little tuning. In particular, for BERT training, our optimizer enables use of very large batch sizes of 32868 without any degradation of performance.  By increasing the batch size to the memory limit of a TPUv3 Pod, BERT training time can be reduced from 3 days to just 76 minutes (Table \ref{table:results}).
\fi


\iffalse
\begin{abstract}
Training large deep neural networks on massive datasets is  computationally very challenging. One promising approach to tackle this issue is through the use of \emph{large batch} parallel stochastic optimization. However, our understanding of this approach in the context of deep learning is still very limited. Furthermore, the current approaches in this direction are heavily hand-tuned. To this end, we first study a general adaptation strategy to accelerate training of deep neural networks using large mini-batches. Using this strategy, we develop a new layer-wise adaptive large batch optimization technique called $\lamb$. We also provide a formal convergence analysis of $\lamb$ as well as the previous published layerwise optimizer $\lars$, showing convergence to a stationary point in general nonconvex settings. Our empirical results demonstrate the superior performance of $\lamb$ for BERT and ResNet-50 training.  In particular, for BERT training, our optimizer enables use of very large batches sizes of 32868; thereby, requiring just 8599 iterations to train (as opposed to 1 million iterations in the original paper).  By increasing the batch size to the memory limit of a TPUv3 Pod,  BERT training time can be reduced from 3 days to 76 minutes (Table \ref{table:results}). Finally, we demonstrate that $\lamb$ outperforms previous large-batch training algorithms for ResNet-50 on ImageNet; obtaining state-of-the-art performance in just a few minutes.
%\textcolor{blue}{An implementation of LAMB can be found at https://github.com/xxx.} 
%The training scripts of BERT and ResNet-50 are available upon request.
More results are in the appendix of this paper.
\end{abstract}
\fi

\section{Introduction}
%!TEX root = nips_2018.tex

With the advent of large scale datasets, training large deep neural networks, even using computationally efficient optimization methods like Stochastic gradient descent $(\sgd)$, has become particularly challenging. For instance, training state-of-the-art deep learning models like \bert and ResNet-50 takes 3 days on 16 TPUv3 chips and 29 hours on 8 Tesla P100 gpus respectively \citep{devlin2018bert,he2016deep}. Thus, there is a growing interest to develop optimization solutions to tackle this critical issue. The goal of this paper is to investigate and develop optimization techniques to accelerate training large deep neural networks, mostly focusing on approaches based on variants of \sgd. 

Methods based on $\sgd$ iteratively update the parameters of the model by moving them in a scaled (negative) direction of the gradient calculated on a minibatch.   However, $\sgd$'s scalability is limited by its inherent sequential nature. Owing to this limitation, traditional approaches to improve \sgd training time in the context of deep learning largely resort to distributed asynchronous setup~\citep{dean2012large,recht2011hogwild}. However,  the implicit staleness introduced due to the asynchrony limits the parallelization of the approach, often leading to degraded performance. The feasibility of computing gradient on \emph{large minibatches} in parallel due to recent hardware advances has seen the resurgence of simply using synchronous \sgd with large minibatches as an alternative to asynchronous \sgd. However, na\"ively increasing the batch size typically results in degradation of generalization performance and reduces computational benefits \citep{goyal2017accurate}.

Synchronous $\sgd$ on large minibatches benefits from reduced variance of the stochastic gradients used in $\sgd$. This allows one to use much larger learning rates in $\sgd$, typically of the order square root of the minibatch size. Surprisingly, recent works have demonstrated that up to certain minibatch sizes, linear scaling of the learning rate with minibatch size can be used to further speed up the training \cite{goyal2017accurate}. These works also elucidate two interesting aspects to enable the use of linear scaling in large batch synchronous $\sgd$: (i) linear scaling of learning rate is harmful during the initial phase; thus, a hand-tuned warmup strategy of slowly increasing the learning rate needs to be used initially, and (ii) linear scaling of learning rate can be detrimental beyond a certain batch size. Using these tricks, \cite{goyal2017accurate} was able to drastically reduce the training time of ResNet-50 model from 29 hours to 1 hour using a batch size of 8192. While these works demonstrate the feasibility of this strategy for reducing the wall time for training large deep neural networks, they also highlight the need for an adaptive learning rate mechanism for large batch learning. 

Variants of $\sgd$ using layerwise adaptive learning rates have been recently proposed to address this problem. The most successful in this line of research is the $\lars$ algorithm \citep{you2017scaling}, which was initially proposed for training $\resnet$. Using $\lars$, ResNet-50 can be trained on ImageNet in just a few minutes! However, it has been observed that its performance gains are \emph{not} consistent across tasks. For instance, $\lars$ performs poorly for attention models like $\bert$. Furthermore, theoretical understanding of the adaptation employed in $\lars$ is largely missing. To this end, we study and develop new approaches specially catered to the large batch setting of our interest.

{\bf Contributions.} More specifically, we make the following main contributions in this paper.

\begin{itemize}
\item Inspired by $\lars$, we investigate a general adaptation strategy specially catered to large batch learning and provide intuition for the strategy.
\item Based on the adaptation strategy, we develop a new optimization algorithm (\lamb) for achieving adaptivity of learning rate in $\sgd$. Furthermore, we provide convergence analysis for both $\lars$ and $\lamb$ to achieve a stationary point in nonconvex settings. We highlight the benefits of using these methods for large batch settings.
\item We demonstrate the strong empirical performance of $\lamb$ across several challenging tasks. Using $\lamb$ we scale the batch size in training $\bert$ to more than 32k without degrading the performance; thereby, cutting the time down from 3 days to 76 minutes. Ours is the first work to reduce $\bert$ training wall time to less than couple of hours.
\item We also demonstrate the efficiency of $\lamb$ for training state-of-the-art image classification models like $\resnet$. To the best of our knowledge, ours is first adaptive solver that can achieve state-of-the-art accuracy for $\resnet$-50 as adaptive solvers like Adam fail to obtain the accuracy of $\sgd$ with momentum for these tasks.
\end{itemize}

\subsection{Related Work}

The literature on optimization for machine learning is vast and hence, we restrict our attention to the most relevant works here. Earlier works on large batch optimization for machine learning mostly focused on convex models, benefiting by a factor of square root of batch size using appropriately large learning rate. Similar results can be shown for nonconvex settings wherein using larger minibatches improves the convergence to stationary points; albeit at the cost of extra computation. However, several important concerns were raised with respect to generalization and computational performance in large batch nonconvex settings. It was observed that training with extremely large batch was difficult~\citep{keskar2016large, hoffer2017train}. Thus, several prior works carefully hand-tune training hyper-parameters, like learning rate and momentum, to avoid degradation of generalization performance \citep{goyal2017accurate, li2017scaling, you2018imagenet, shallue2018measuring}. 

\citep{krizhevsky2014one} empirically found that simply scaling the learning rate linearly with respect to batch size works better up to certain batch sizes. To avoid optimization instability due to linear scaling of learning rate, \citet{goyal2017accurate} proposed a highly hand-tuned learning rate which involves a warm-up strategy that gradually increases the LR to a larger value and then switching to the regular LR policy (e.g. exponential or polynomial decay). Using LR warm-up and linear scaling, \citet{goyal2017accurate} managed to train $\resnet$-50 with batch size 8192 without loss in generalization performance. However, empirical study \citep{shallue2018measuring} shows that learning rate scaling heuristics with the batch size do not hold across all problems or across all batch sizes.

More recently, to reduce hand-tuning of hyperparameters, adaptive learning rates for large batch training garnered significant interests. Several recent works successfully scaled the batch size to large values using adaptive learning rates without degrading the performance, thereby, finishing $\resnet$-50 training on ImageNet in a few minutes \citep{you2018imagenet,iandola2016firecaffe,codreanu2017scale,akiba2017extremely,jia2018highly,smith2017don,martens2015optimizing,devarakonda2017adabatch,mikami2018imagenet,osawa2018second,you2019large,yamazaki2019yet}.
To the best of our knowledge, the fastest training result for $\resnet$-50 on ImageNet is due to \cite{ying2018image}, who achieve 76+\% top-1 accuracy. By using the $\lars$ optimizer and scaling the batch size to 32K  on a TPUv3 Pod, \citet{ying2018image} was able to train $\resnet$-50 on ImageNet in 2.2 minutes. However, it was empirically observed that none of these performance gains hold in other tasks such as BERT training (see Section~\ref{sec:experiments}).  


\iffalse
\subsection{Related Work}

The literature on optimization for machine learning is vast and hence, we restrict our attention to the works on large batch settings that are most relevant to our paper. Earlier works on large batch optimization for machine learning mostly focused on convex models. It is known that for general stochastic convex objective functions, the convergence of $\sgd$ with minibatch $b$ is $O(1/\sqrt{bT} + 1/T)$.  If a more complex optimization problem is solved in each iteration, the convergence rate can be improved to $O(1/\sqrt{bT})$, which improves when batch size $b$ is large. Similar results can be shown for nonconvex settings wherein using larger minibatches improves the convergence to stationary points; albeit at the cost of extra computation. However, several important concerns were raised with respect to generalization and computational performance in large batch nonconvex settings. It was observed that training with extremely large batch was difficult~\citep{keskar2016large, hoffer2017train}. The researchers need to carefully tune training hyper-parameters, like learning rate and momentum, to avoid losing test accuracy \citep{goyal2017accurate, li2017scaling, you2018imagenet, shallue2018measuring}. 

\citet{krizhevsky2014one} introduced some practical schemes for training with large batches. One important rule is to increase the LR (learning rate) by $\sqrt{b}$ when batch size is scaled by $b$ since  the variance of the gradient estimation decreases by a factor of $b$. In practice, \citep{krizhevsky2014one} found that  linear scaling works better up to certain batch sizes. To avoid optimization instability due to high learning rate, \citet{goyal2017accurate} proposed to use a highly hand-tuned  learning rate warm-up strategy which starts with a small LR and then gradually increases the LR to a larger value. After warm-up period (usually a few epochs) one switches to the regular LR policy (multi-steps, exponential or polynomial decay etc). Using LR warm-up and linear scaling, \citet{goyal2017accurate} managed to train $\resnet$-50 with batch size 8192 without loss in test accuracy. However, empirical study \citep{shallue2018measuring} shows that learning rate scaling heuristics with the batch size do not hold across all problems or across all batch sizes.

More recently, to reduce hand-tuning of hyperparameters, adaptive learning rates for large batch training garnered significant interests. Several recent works successfully scaled the batch size to large values using adaptive learning rates without degrading the performance, thereby, finishing $\resnet$-50 training on ImageNet in a few minutes \citep{you2018imagenet,iandola2016firecaffe,codreanu2017scale,akiba2017extremely,jia2018highly,smith2017don,martens2015optimizing,devarakonda2017adabatch,mikami2018imagenet,osawa2018second,you2019large,yamazaki2019yet}.
To the best of our knowledge, the fastest training result for $\resnet$-50 on ImageNet is due to \cite{ying2018image}, who achieve 76+\% top-1 accuracy. By using the $\lars$ optimizer and scaling the batch size to 32K  on a TPUv3 Pod, \citet{ying2018image} was able to train $\resnet$-50 on ImageNet in 2.2 minutes. 

\fi

\section{Preliminaries}
%!TEX root = nips_2018.tex

\paragraph{Notation.}  
For any vector $x_t \in \mathbb{R}^d$, either $x_{t,j}$ or $[x_t]_{j}$ are used to denote its $j^{\text{th}}$ coordinate where $j \in [d]$.  Let $\mathbb{I}$ be the $d \times d$ identity matrix, and let $\mathbb{I} = [\mathbb{I}_1, \mathbb{I}_2,...,\mathbb{I}_h]$ be its decomposition into column submatrices $\mathbb{I}_i = d \times d_h$. For $x \in \mathbb{R}^d$, let $x^{(i)}$ be the block of variables corresponding to the columns of $I_i$ i.e., $x^{(i)} = \mathbb{I}_i^\top x \in \mathbb{R}^{d_i}$  for
$i = \{1, 2, \cdots ,h\}$. For any function $f:\mathbb{R}^d \rightarrow \mathbb{R}$, we use $\nabla_i f(x)$ to denote the gradient with respect to $x^{(i)}$. For any vectors $u, v \in \mathbb{R}^d$, we use $u^2$ and $u/v$ to denote elementwise square and division operators respectively.
We use $\|.\|$ and $\|.\|_1$ to denote $l_2$-norm and $l_1$-norm of a vector respectively.

We start our discussion by formally stating the problem setup.  In this paper, we study nonconvex stochastic optimization problems of the form
\begin{align}
\label{eq:1}
\min_{x \in \mathbb{R}^d} f(x) := \mathbb{E}_{s \sim \mathbb{P}}[\ell(x, s)] + \frac{\lambda}{2} \|x\|^2,
\end{align}
where $\ell$ is a smooth (possibly nonconvex) function and $\mathbb{P}$ is a probability distribution on the domain $\mathcal{S} \subset \mathbb{R}^k$. 
Here, $x$ corresponds to model parameters, $\ell$ is the loss function and $\mathbb{P}$ is an unknown data distribution. 

We assume function $\ell(x)$ is $L_i$-\emph{smooth} with respect to $x^{(i)}$, i.e.,  there exists a constant $L_i$ such that
\begin{equation}
\label{eq:l-const}
  \|\nabla_i \ell(x, s)-\nabla_i \ell(y, s)\| \le L_i\|x^{(i)}-y^{(i)}\|,\quad\forall\ x, y \in \reals^d, \text{ and  } s \in \mathcal{S},
\end{equation}
for all $i \in [h]$. We use $L = (L_1, \cdots, L_h)^\top$ to denote the $h$-dimensional vector of Lipschitz constants. 
We use $L_\infty$ and $L_{avg}$ to denote $\max_i L_i$ and $\sum_i \tfrac{L_i}{h}$ respectively.  We assume the following bound on the variance in stochastic gradients: $\mathbb{E}\|\nabla_i \ell(x, s) - \nabla_i f(x)\|^2 \leq \sigma_i^2$ for all $x \in \mathbb{R}^d$ and $i \in [h]$.    
Furthermore, we also assume $\mathbb{E}\|[\nabla \ell(x, s)]_i - [\nabla f(x)]_i\|^2 \leq \tilde{\sigma}_i^2$ for all $x \in \mathbb{R}^d$ and $i \in [d]$.  
We use $\sigma = (\sigma_1, \cdots, \sigma_h)^\top$ and $\tilde{\sigma} = (\tilde{\sigma}_1, \cdots, \tilde{\sigma}_d)^\top$ to denote the vectors of standard deviations of stochastic gradient per layer and per dimension respectively.  
Finally, we assume that the gradients are bounded i.e., $[\nabla l(x,s)]_j \leq G$ for all $i \in [d]$, $x \in \mathbb{R}^d$ and $s \in \mathcal{S}$. 
Note that such assumptions are typical in the analysis of stochastic first-order methods (cf. \citep{Ghadimi13,Ghadimi14}). 

Stochastic gradient descent (\sgd) is one of the simplest first-order algorithms for solving problem in Equation~\ref{eq:1}. The update at the $t^{\text{th}}$ iteration of $\sgd$ is of the following form:
\begin{equation}
\tag{$\sgd$}
x_{t+1} = x_{t} - \eta_t  \frac{1}{|\mathcal{S}_t|} \sum_{s_t \in \mathcal{S}_t} \nabla \ell(x_t, s_t) + \lambda x_t,
\end{equation}
where $S_t$ is set of $b$ random samples drawn from the  distribution $\mathbb{P}$. For very large batch settings, the following is a well-known result for $\sgd$.
\begin{theorem}[\citep{ghadimi2013stochastic}]
With large batch $b=T$ and using appropriate learning rate, we have the following for the iterates of $\sgd$:
\begin{align*}
\mathbb{E}\left[\|\nabla f(x_a)\|^2\right] \leq O\left(\frac{(f(x_1) - f(x^*))L_{\infty}}{T} + \frac{\|\sigma \|^2}{T}\right).
\end{align*}
where $x^*$ is an optimal solution to the problem in \eqref{eq:1} and $x_a$ is an iterate uniformly randomly chosen from $\{x_1, \cdots, x_T\}$.
\label{thm:sgd-conv}
\end{theorem}
However, tuning the learning rate $\eta_t$ in $\sgd$, especially in large batch settings, is difficult in practice. Furthermore, the dependence on $L_\infty$ (the maximum of smoothness across dimension) can lead to significantly slow convergence. In the next section, we discuss algorithms to circumvent this issue. 

\section{Algorithms}
%!TEX root = nips_2018.tex

In this section, we first discuss a general strategy to adapt the learning rate in large batch settings.  Using this strategy, we discuss two specific algorithms in the later part of the section. Since our primary focus is on deep learning, our discussion is centered around training a $h$-layer neural network.

{\bf General Strategy.} Suppose we use an iterative \emph{base} algorithm $\mathcal{A}$ (e.g. $\sgd$ or $\adam$) in the small batch setting with the following layerwise update rule:
\begin{align*}
x_{t+1} = x_t + \eta_t u_t, 
\end{align*} 
where $u_t$ is the update made by $\mathcal{A}$ at time step $t$. We propose the following two changes to the update for large batch settings:
\begin{enumerate}
\item The update is normalized to unit $l_2$-norm. This is ensured by modifying the update to the form $u_t/\|u_t\|$. Throughout this paper, such a normalization is done layerwise i.e., the update for each layer is ensured to be unit $l_2$-norm.
\item The learning rate is scaled by $\phi(\|x_t\|)$ for some function $\phi:\mathbb{R}^{+} \rightarrow \mathbb{R}^+$. Similar to the normalization, such a scaling is done layerwise.
\end{enumerate}
Suppose the base algorithm $\mathcal{A}$ is $\sgd$, then the modification results in the following update rule:
\begin{align}
x_{t+1}^{(i)} = x_t^{(i)} - \eta_t \frac{\phi(\|x_t^{(i)}\|)}{\|g_t^{(i)}\|} g_t^{(i)} ,
\end{align}
for all layers $i \in [h]$ and where $x^{(i)}_t$ and $g^{(i)}_t$ are the parameters and the gradients of the $i^{\text{th}}$ layer at time step $t$.  The normalization modification is similar to one typically used in normalized gradient descent except that it is done layerwise. Note that the modification leads to a biased gradient update; however, in large-batch settings, it can be shown that this bias is small.  It is intuitive  that such a normalization provides robustness to exploding gradients (where the gradient can be arbitrarily large) and plateaus (where the gradient can be arbitrarily small). Normalization of this form essentially ignores the size of the gradient and is particularly useful in large batch settings where the direction of the gradient is largely preserved.

The scaling term involving $\phi$ ensures that the norm of the update is of the same order as that of the parameter. We found that this typically ensures faster convergence in deep neural networks. In practice, we observed that a simple function of $\phi(z) = \min\{\max\{z, \gamma_l\}, \gamma_u\}$ works well. It is instructive to consider the case where $\phi(z) = z$. In this scenario, the overall change in the learning rate is $\tfrac{\|x_t^{(i)}\|}{\|g_t^{(i)}\|}$ , which can also be interpreted as an estimate on the inverse of Lipschitz constant of the gradient (see~\eqref{eq:l-const}). We now discuss different instantiations of the strategy discussed above. 
In particular, we focus on two algorithms: $\lars$ (\ref{Subsection:lars}) and the proposed method, $\lamb$ (\ref{Subsection:lamb}).

\subsection{$\lars$ Algorithm}
\label{Subsection:lars}

The first instantiation of the general strategy is $\lars$ algorithm \citep{you2017scaling}, which is obtained by using momentum optimizer as the base algorithm $\mathcal{A}$ in the framework. $\lars$ was earlier proposed for large batch learning for $\resnet$ on ImageNet. In general, it is observed that the using (heavy-ball) momentum, one can reduce the variance in the stochastic gradients at the cost of little bias. The pseudocode for $\lars$ is provide in Algorithm~\ref{alg:lars}.

\begin{figure}
\begin{minipage}[b]{.48\textwidth}
\begin{algorithm}[H]\small
	\caption{$\lars$}
	\label{alg:lars}
	\begin{algorithmic}
		\STATE {\bfseries Input:} $x_1 \in \mathbb{R}^d$, learning rate $\{\eta_t\}_{t=1}^T$, parameter $0 < \beta_{1} < 1$, scaling function $\phi$, $\epsilon > 0$
		\STATE Set $m_{0} = 0$
		\FOR{$t=1$ {\bfseries to} $T$}
		\STATE Draw b samples $S_t$ from $\mathbb{P}$
        \STATE Compute $g_t = \frac{1}{|\mathcal{S}_t|} \sum_{s_t \in \mathcal{S}_t}\nabla \ell(x_t, s_t)$
        \STATE $m_{t} = \beta_{1} m_{t-1} + (1 - \beta_{1}) (g_{t} + \lambda x_t)$
		\STATE $x_{t+1}^{(i)} = x_{t}^{(i)} - \eta_t \frac{\phi(\|x_t^{(i)}\|)}{\|m_t^{(i)}\|} m_t^{(i)}$ for all $i \in [h]$
		\ENDFOR
	\end{algorithmic}
\end{algorithm}
\end{minipage}\hfill% This must go next to `\end{minipage}`
\begin{minipage}[b]{.5\textwidth}
\begin{algorithm}[H]\small
	\caption{$\lamb$}
	\label{alg:lamb}
	\begin{algorithmic}
		\STATE {\bf Input:} $x_1 \in \mathbb{R}^d$, learning rate $\{\eta_t\}_{t=1}^T$,  parameters $0 < \beta_{1}, \beta_2 < 1$, scaling function $\phi$, $\epsilon > 0$
		\STATE Set $m_{0} = 0$, $v_{0} = 0$
		\FOR{$t=1$ {\bf to} $T$}
		\STATE Draw b samples $S_t$ from $\mathbb{P}$.
        \STATE Compute $g_t = \frac{1}{|\mathcal{S}_t|} \sum_{s_t \in \mathcal{S}_t}\nabla \ell(x_t, s_t)$.
		\STATE  $m_{t} = \beta_{1} m_{t-1} + (1 - \beta_{1}) g_{t}$ 
		\STATE  $v_{t} = \beta_{2} v_{t-1} + (1 - \beta_{2}) g_{t}^2$
		\STATE $m_t = m_t/(1 - {\beta}_1^t)$ 
        \STATE $v_t = v_t/(1 - {\beta}_2^t)$
		\STATE Compute ratio $r_t = \frac{m_t}{\sqrt{v_t} + \epsilon}$
		\STATE $x_{t+1}^{(i)} = x_{t}^{(i)} - \eta_t \frac{\phi(\|x_t^{(i)}\|)}{\|r_t^{(i)} + \lambda x_t^{(i)}\|} (r_t^{(i)} + \lambda x_t^{(i)})$
		\ENDFOR
	\end{algorithmic}
\end{algorithm}
\end{minipage}
\end{figure}

We now provide convergence analysis for $\lars$ in general nonconvex setting stated in this paper. For the sake of simplicity, we analyze the case where $\beta_1 = 0$ and $\lambda = 0$ in Algorithm~\ref{alg:lars}. However, our analysis should extend to the general case as well. We will defer all discussions about the convergence rate to the end of the section.

\begin{theorem}
\label{thm:lars-conv}
Let $\eta_t = \eta = \sqrt{\tfrac{2(f(x_1) - f(x^*))}{\alpha_u^2 \|L\|_1 T}}$ \ for all $t \in [T]$, $b=T$, $\alpha_l \leq \phi(v) \leq \alpha_u$ for all $v > 0$ where $\alpha_l, \alpha_u > 0$. Then for $x_t$ generated using $\lars$ (Algorithm~\ref{alg:lars}), we have the following bound
\begin{align*}
\left(\mathbb{E}\left[\frac{1}{\sqrt{h}}\sum_{i=1}^h \|\nabla_i f(x_a)\|\right]\right)^2 \leq O\left(\frac{(f(x_1) - f(x^*))L_{avg}}{T} + \frac{\|\sigma \|^2_1}{Th}\right), 
\end{align*}
where $x^*$ is an optimal solution to the problem in \eqref{eq:1} and $x_a$ is an iterate uniformly randomly chosen from $\{x_1, \cdots, x_T\}$.
\end{theorem}

\subsection{$\lamb$ Algorithm}
\label{Subsection:lamb}

The second instantiation of the general strategy is obtained by using $\adam$ as the base algorithm $\mathcal{A}$. $\adam$ optimizer is popular in deep learning community and has shown to have good performance for training state-of-the-art language models like $\bert$. Unlike $\lars$, the adaptivity of $\lamb$ is two-fold: (i) per dimension normalization with respect to the square root of the second moment used in $\adam$ and (ii) layerwise normalization  obtained due to layerwise adaptivity. The pseudocode for $\lamb$ is provided in Algorithm~\ref{alg:lamb}. When $\beta_1 = 0$ and $\beta_2 = 0$, the algorithm reduces to be Sign \sgd where the learning rate is scaled by square root of the layer dimension \citep{signsgd}.

The following result provides convergence rate for $\lamb$ in general nonconvex settings. Similar to the previous case, we focus on the setting where $\beta_1 = 0$ and $\lambda = 0$. As before, our analysis extends to the general case; however, the calculations become messy.

\begin{theorem}
\label{thm:lamb-conv}
Let $\eta_t = \eta = \sqrt{\tfrac{2(f(x_1) - f(x^*))}{\alpha_u^2 \|L\|_1 T}}$ \ for all $t \in [T]$, $b = T$, $d_i = d/h$ for all $i \in [h]$, and $\alpha_l \leq \phi(v) \leq \alpha_u$ for all $v > 0$ where $\alpha_l, \alpha_u > 0$. Then for $x_t$ generated using $\lamb$ (Algorithm~\ref{alg:lamb}), we have the following bounds:
\begin{enumerate}
    \item When $\beta_2 = 0$, we have
    \begin{align*}
\left(\mathbb{E}\left[\frac{1}{\sqrt{d}}\|\nabla f(x_a)\|_1\right]\right)^2 &\leq  O\left(\frac{(f(x_1) - f(x^*))L_{avg}}{T} + \frac{\|\tilde{\sigma}\|^2_1}{Th}\right),
\end{align*}

    \item When $\beta_2 > 0$, we have
    \begin{align*}
 \mathbb{E}[\|\nabla f(x_a)\|^2] &\leq O\left(\sqrt{\frac{G^2 d}{h(1 - \beta_2)}} \times \left[\sqrt{\frac{2(f(x_1) - f(x^*))\|L\|_1}{T}} + \frac{\|\tilde{\sigma}\|_1}{\sqrt{T}} \right]\right),
\end{align*}

\end{enumerate}
where $x^*$ is an optimal solution to the problem in \eqref{eq:1} and $x_a$ is an iterate uniformly randomly chosen from $\{x_1, \cdots, x_T\}$.
\end{theorem}

{\bf Discussion on convergence rates.} We first start our discussion with the comparison of convergence rate of $\lars$ with that of $\sgd$ (Theorem~\ref{thm:sgd-conv}). The convergence rates of $\lars$ and $\sgd$ differ in two ways: (1) the convergence criterion is $(\mathbb{E}[\sum_{i=1}^h \|\nabla_i f\|])^2$ as opposed to $\mathbb{E}[\|\nabla f\|^2]$ in $\sgd$ and (2) the dependence on $L$ and $\sigma$ in the convergence rate. Briefly, the convergence rate of $\lars$ is better than $\sgd$ when the gradient is denser than curvature and stochasticity. This convergence rate comparison is similar in spirit to the one obtained in \citep{signsgd}. Assuming that the convergence criterion in Theorem~\ref{thm:sgd-conv} and Theorem~\ref{thm:lars-conv} is of similar order (which happens when gradients are fairly dense), convergence rate of $\lars$ and $\lamb$ depend on $L_{avg}$ instead of $L_\infty$ and are thus, significantly better than that of $\sgd$. A more quantitative comparison is provided in Section~\ref{sec:conv-compare} of the Appendix. The comparison of $\lamb$ (with $\beta_2$ = 0) with $\sgd$ is along similar lines. We obtain slightly worse rates for the case where $\beta_2 > 0$; although, we believe that its behavior should be better than the case $\beta_2 = 0$. We leave this investigation to future work. 

\section{Experiments}
\label{sec:experiments}
%\section{Experimental Results}
We now present empirical results comparing $\lamb$ with existing optimizers on two important large batch training tasks: $\bert$ and $\resnet$-50 training. 
%In the later part of the section, we also show the performance of $\lamb$ on a few small tasks involving CIFAR and MNIST datasets. 
We also compare $\lamb$ with existing optimizers for small batch size ($<1K$) and small dataset (e.g. CIFAR, MNIST) (see Appendix).

{\bf Experimental Setup. } To demonstrate its robustness, we use very minimal hyperparameter tuning for the $\lamb$ optimizer. Thus, it is possible to achieve better results by further tuning the hyperparameters. The parameters $\beta_1 $ and $\beta_2$ in Algorithm~\ref{alg:lamb} are set to $0.9$ and $0.999$ respectively in all our experiments; we only tune the learning rate. We use a polynomially decaying learning rate of $\eta_t = \eta_0 \times (1 - t/T)$ in Algorithm~\ref{alg:lamb}), which is the same as in $\bert$ baseline. This setting also works for all other applications in this paper.
Furthermore, for $\bert$ and $\resnet$-50  training, we did not tune the hyperparameters of $\lamb$ while increasing the batch size. We use the square root of LR scaling rule to automatically adjust learning rate and linear-epoch warmup scheduling. %\textcolor{red}{Yang: explain what square root LR scaling is and linear epoch warmup is. Please provide details and connect them to the parameters of Algorithm~\ref{alg:lamb}} %\textcolor{red}{Yang: describe the whole experimental setup here e.g. TPU etc}. 
We use TPUv3 in all the experiments.
A TPUv3 Pod has 1024 chips and can provide more than 100 petaflops performance for mixed precision computing.
%The details can be found in Tables \ref{table:hyper_parameters} and \ref{table:resnet_hyper_parameters} in Appendix.
%\textcolor{blue}{Please show some data here.}
To make sure we are comparing with solid baselines, we use grid search to tune the hyper-parameters for 
%and an internal hyper-parameter tuning tool for  
$\adam$, $\adagrad$, $\adamw$ ($\adam$ with weight decay), and $\lars$. We also tune weight decay for $\adamw$. All the hyperparameter tuning settings are reported in the Appendix. %{\color{red}(Cho: can we say using an internal tool?)}
%{\color{blue}(Yang: internal means Google's. Please feel free to remove this sentence.)}
Due to space constraints, several experimental details are relegated to the Appendix.


%The LAMB optimizer does not expect the users to heavily tune the hyper-parameters.
%Like Adam and AdamW, the users only need to input the learning rate.
%We conduct a comparison between LAMB and existing optimizers.
%Adam, AdamW, and LAMB have three additional hyper-parameters (${\beta}_1, {\beta}_2, \epsilon$). 
%We did not tune these three hyper-parameters. We just use the default values. We only tune the learning rate. If the application uses learning rate warmup, we also tune the warmup epochs. 
%To make sure we are comparing a solid baseline, we use grid search and an internal hyper-parameter tuning tool for Adam, Adagrad, AdamW, and LARS. We also tune weight decay for AdamW. All the hyper-parameter tuning spaces are in the appendix of this paper.
%Besides BERT, LAMB performs well on a wide range of applications in our experiments.
%Bulatov et al. reported LAMB was also successfully used in large-scale transformer models\footnote{https://medium.com/south-park-commons/scaling-transformer-xl-to-128-gpus-85849508ec35} recently.
%\subsection{\textcolor{blue}{Implementation Details}}
%We use Jacob Devlin's implementation on BERT's github\footnote{https://github.com/google-research/bert/blob/master/optimization.py} as the AdamW baseline (as of March 1st, 2019).
%\textcolor{blue}{\textcolor{blue}{Remove the learning rate rampup in Adam and AdamW because warmup essentially has the same effect.}}

\subsection{$\bert$ Training}
%\subsubsection{Fair Comparison}
%\textcolor{blue}{Based on our experience, we can trust validation loss.
%For example, in ImageNet training with ResNet at 8K batch size, a lower validation loss actually leads to worse accuracy!}
%\textcolor{blue}{sqrt root decay (poly power = 0.5) should be part of the algorithm because I did not find anything is better than that.}
We first discuss empirical results for speeding up $\bert$ training. For this experiment, we use the same dataset as \cite{devlin2018bert}, which is a concatenation of Wikipedia and BooksCorpus with 2.5B and 800M words respectively. We specifically focus on the SQuAD task\footnote{https://rajpurkar.github.io/SQuAD-explorer/} in this paper. 
%Stanford Question Answering Dataset (SQuAD) is a reading comprehension dataset which contains questions posed by crowdworkers on a set of Wikipedia articles, the answer to which is a segment of text from the provided reading passage. 
The F1 score on SQuAD-v1 is used as the accuracy metric in our experiments. All our comparisons are with respect to the baseline $\bert$ model by \cite{devlin2018bert}. To train $\bert$, \citet{devlin2018bert} first train the model for 900k iterations using a sequence length of 128 and then switch to a sequence length of 512 for the last 100k iterations. This results in a training time of around 3 days on 16 TPUv3 chips. The baseline $\bert$ model\footnote{Pre-trained BERT model can be downloaded  from https://github.com/google-research/bert} achieves a F1 score of 90.395. To ensure a fair comparison, we follow the same SQuAD fine-tune procedure of~\cite{devlin2018bert} without modifying any configuration (including number of epochs and hyperparameters). As noted earlier, we could get even better results by changing the fine-tune configuration. For instance, by just slightly changing the learning rate in the fine-tune stage, we can obtain a higher F1 score of 91.688 for the batch size of 16K using $\lamb$. We report a F1 score of 91.345 in Table \ref{table:results}, which is the score obtained for the untuned version. Below we describe two different training choices for training $\bert$ and discuss the corresponding speedups.

\begin{table}[ht]
\renewcommand{\arraystretch}{1.3}
\caption{ We use the F1 score on SQuAD-v1 as the accuracy metric. The baseline F1 score is the score obtained by the pre-trained model ($\bert$-Large) provided on $\bert$'s public repository (as of February 1st, 2019). We use TPUv3s in our experiments. We use the same setting as the baseline: the first 9/10 of the total epochs used a sequence length of 128 and the last 1/10 of the total epochs used a sequence length of 512. All the experiments run the same number of epochs. Dev set means the test data. It is worth noting that we can achieve better results by manually tuning the hyperparameters. The data in this table is collected from the untuned version.}
\centering
 
\begin{tabular}{|c|c|c|c|c|c|}
\hline
Solver & batch size & steps & F1 score on dev set & TPUs & Time\\
\hline
\hline
Baseline & 512 & 1000k & 90.395 & 16 & 81.4h\\
\hline
$\lamb$ & 512 & 1000k & 91.752 & 16 & 82.8h\\
\hline
$\lamb$ & 1k & 500k & 91.761 & 32 & 43.2h\\
\hline
$\lamb$ & 2k & 250k & 91.946 & 64 & 21.4h\\
\hline
$\lamb$ & 4k & 125k & 91.137 & 128 & 693.6m\\
\hline
$\lamb$ & 8k & 62500 & 91.263 & 256 & 390.5m\\
\hline
$\lamb$ & 16k & 31250 & 91.345 & 512 & 200.0m\\
\hline
$\lamb$ & 32k & 15625 & 91.475 & 1024 & 101.2m\\
\hline
\hline
$\lamb$ & 64k/32k & 8599 & 90.584 & 1024 & 76.19m\\
\hline
%\textcolor{blue}{Our Method} & 128k/32k & --- & 89.559 & 1024 TPUs & ---\\
%\hline
\end{tabular}
\label{table:results}
\end{table}



%We make sure all the fine-tune comparisons use the same hyper-parameters.

%People with limited hardware resource should be able to reproduce our results. We can use a large batch size on a single machine regardless of the number of chips or amount of available memory. We can simply iterate over multiple batches and accumulates the resulting gradients before committing a weight update.
%We confirm this simulation can reproduce the same results as the distributed implemenation.

%\paragraph{Regular Training using $\lamb$} 
%\textcolor{red}{Yang: What exactly needs to go here?}
%The Tensor Processing Units \cite{jouppi2017datacenter} are powerful computational hardware for floating-point operations.
%The baseline uses the  Wikipedia and BooksCorpus datasets \cite{zhu2015aligning} in the pretraining.


For the first choice, we maintain the same training procedure as the baseline except for changing the training optimizer to $\lamb$. We run with the same number of epochs as the baseline but with batch size scaled from 512 to 32K. The choice of 32K batch size (with sequence length 512) is mainly due to memory limits of TPU Pod. Our results are shown in Table \ref{table:results}. By using the $\lamb$ optimizer, we are able to achieve a F1 score of 91.460 in 15625 iterations for a batch size of 32768 (14063 iterations for sequence length 128 and 1562 iterations for sequence length 512).
With 32K batch size, we reduce $\bert$ training time from 3 days to around 100 minutes. 
We achieved 49.1 times speedup by 64 times computational resources (76.7\% efficiency).
We consider the speedup is great because we use the synchronous data-parallelism. 
There is a communication overhead coming from transferring of the gradients over the interconnect.
%The gradients have the same size of the trained models.
For $\resnet$-50, researchers are able to achieve 90\% scaling efficiency because $\resnet$-50 has much fewer parameters (\# parameters is equal to \#gradients) than $\bert$ (25 million versus 300 million).
%\textcolor{blue}{The optimizer's TensorFlow code and the trained model can be found at http://XXX}
%\textcolor{blue}{It is worth noting that we did not change any configuration in the SQuAD fine-tune stage. We used the same number of epochs as the baseline in the finetune. Otherwise, the comparison will not be fair.}
%\paragraph{Mixed-Batch Training using $\lamb$}

To obtain further improvements, we use the {\bf Mixed-Batch Training} procedure with $\lamb$. 
Recall that $\bert$ training involves two stages: the first 9/10 of the total epochs use a sequence length of 128, while the last 1/10 of the total epochs use a sequence length of 512. 
For the second stage training, which involves a longer sequence length, due to memory limits, a maximum batch size of only 32768 can be used on a TPUv3 Pod. However, we can potentially use a larger batch size for the first stage because of a shorter sequence length. 
%In particular, the batch size can be increased to 65536 for the first stage. 
In particular, the batch size can be increased to 131072 for the first stage. 
However, we did not observe any speedup by increasing the batch size from 65536 to 131072 for the first stage, thus, we restrict the batch size to 65536 for this stage. 
%Prior to this work, \citet{smith2017don} also studied the mixed-batch training strategy. However, they increase the batch size during training while we decrease the batch size. 
By using this strategy, we are able to make full utilization of the hardware resources throughout the training procedure. 
Increasing the batch size is able to warm-up and stabilize the optimization process \citep{smith2017don}, but decreasing the batch size brings chaos to the optimization process and can cause divergence.
In our experiments, we found a technique that is useful to stabilize the second stage optimization.
Because we switched to a different optimization problem, it is necessary to re-warm-up the optimization.
Instead of decaying the learning rate at the second stage, we ramp up the learning rate from zero again in the second stage (re-warm-up).
As with the first stage, we decay the learning rate after the re-warm-up phase.
With this method, we only need 8599 iterations and finish $\bert$ training in 76 minutes (100.2\% efficiency).


%\subsection{\textcolor{blue}{Reduced-Epochs Training}}
%\textcolor{blue}{Try only using seq=512 to train for 3K iterations}

\paragraph{Comparison with $\adamw$ and $\lars$.}
To ensure that our approach is compared to a solid baseline for the $\bert$ training, we tried three different strategies for tuning $\adamw$: (1) $\adamw$ with default hyperparameters (see \cite{devlin2018bert}) (2) $\adamw$ with the same hyperparameters as $\lamb$, and (3) $\adamw$ with tuned hyperparameters. $\adamw$ stops scaling at the batch size of 16K because it is not able to achieve the target F1 score (88.1 vs 90.4). 
%Table \ref{table:adamw_16k} shows some of the tuning information.  
The tuning information of $\adamw$ is shown in the Appendix.
For 64K/32K mixed-batch training, even after extensive tuning of the hyperparameters, we fail to get any reasonable result with $\adamw$ optimizer. 
%For example, we only got a F1 score of 5.492.
We conclude that $\adamw$ does not work well in large-batch $\bert$ training or is at least hard to tune.
%We also tried using the hyper-parameters of $\lamb$ to AdamW training. However, it also did not produce any reasonable result.
We also observe that $\lamb$ performs better than $\lars$ for all batch sizes (see Table \ref{table:lars_lamb_bert}).

\begin{table}[ht]
\renewcommand{\arraystretch}{1.3}
\caption{ $\lamb$ achieves a higher performance (F1 score) than $\lars$ for all the batch sizes. The baseline achieves a F1 score of 90.390. Thus, $\lars$ stops scaling at the batch size of 16K.}
\centering

\begin{tabular}{|c|c|c|c|c|c|c|c|}
\hline
Batch Size & 512 & 1K & 2K & 4K & 8K & 16K & 32K\\
\hline
\hline
$\lars$ & 90.717 & 90.369 & 90.748 & 90.537 & 90.548 & 89.589 & diverge \\
\hline
$\lamb$ & 91.752 & 91.761 & 91.946 & 91.137 & 91.263 & 91.345 & 91.475 \\
\hline
\end{tabular}
\label{table:lars_lamb_bert}
\end{table}

\subsection{ImageNet Training with ResNet-50.}
ImageNet training with ResNet-50 is an industry standard metric that is being used in MLPerf\footnote{https://mlperf.org/}. 
%According to Stanford Dawnbench \cite{coleman2017dawnbench}, ResNet-50 is the fastest model to achieve 93\% top-5 accuracy for ImageNet classification (as of April 1st, 2019).
The baseline can get 76.3\% top-1 accuracy in 90 epochs \citep{goyal2017accurate}.
All the successful implementations are based on momentum SGD \citep{he2016deep, goyal2017accurate} or $\lars$ optimizer \citep{ying2018image, jia2018highly, mikami2018imagenet, you2018imagenet,yamazaki2019yet}.
Before our study, we did not find any paper reporting a state-of-the-art accuracy achieved by $\adam$, $\adagrad$, or $\adamw$ optimizer.
In our experiments, even with comprehensive hyper-parameter tuning, $\adagrad$/$\adam$/$\adamw$ (with batch size 16K) only achieves 55.38\%/66.04\%/67.27\% top-1 accuracy.
After adding learning rate scheme of \cite{goyal2017accurate},
%\textcolor{red}{Yang: please refrain from using name of the author. Instead please provide a reference. This problems occurs at multiple places. Please edit.}, 
the top-1 accuracy of $\adagrad$/$\adam$/$\adamw$ was improved to 72.0\%/73.48\%/73.07\%.
However, they are still much lower than 76.3\%.
The details of the tuning information are in the Appendix.
Table \ref{table:resnet50_acc} shows that $\lamb$ can achieve the target accuracy.
Beyond a batch size of 8K, $\lamb$'s accuracy is higher than the momentum.
$\lamb$'s accuracy is also slightly better than $\lars$.
At a batch size of 32K, $\lamb$ achieves 76.4\% top-1 accuracy while $\lars$ achieves 76.3\%.
At a batch size of 2K, $\lamb$ is able to achieve 77.11\% top-1 accuracy while $\lars$ achieves 76.6\%.

\iffalse
\begin{figure*}[tb]
\vspace{5pt}
\centering
\includegraphics[width=0.88\textwidth]{figs/imagenet_resnet50.png}
\caption{$\lamb$ is able to achieve state-of-the-art accuracy in ImageNet/ResNet-50 training for large-batch training. The performance of momentum solver was reported by \citep{goyal2017accurate}. $\adam +$ means adding the learning rate scheme of \cite{goyal2017accurate} to $\adam$: (1) 5-epoch warmup to stablize the initial stage; and (2) multiply the learning rate by 0.1 at 30th, 60th, and 80th epoch. The target accuracy is around 0.763 \citep{goyal2017accurate}. All the adaptive solvers were comprehensively tuned. The tuning information was in the appendix of this paper.}
\label{fig:resnet50_acc}
\vspace{-10pt}
\end{figure*}
\fi

\begin{table}[ht]
\renewcommand{\arraystretch}{1.3}
\caption{ Top-1 validation accuracy of ImageNet/$\resnet$-50 training at the batch size of 16K (90 epochs). The performance of momentum was reported by \citep{goyal2017accurate}. + means adding the learning rate scheme of \cite{goyal2017accurate} to the optimizer: (1) 5-epoch warmup to stablize the initial stage; and (2) multiply the learning rate by 0.1 at 30th, 60th, and 80th epoch. The target accuracy is around 0.763 \citep{goyal2017accurate}. All the adaptive solvers were comprehensively tuned. The tuning information was in the Appendix.}
\centering

\begin{tabular}{|c|c|c|c|c|c|}
\hline
%Optimizer & $\adagrad$/$\adagrad$+ & $\adam$/$\adam$+ & $\adamw$/$\adamw$+ & momentum & $\lamb$ \\
optimizer & adagrad/adagrad+ & adam/adam+ & adamw/adamw+ & momentum & lamb \\
\hline
\hline
Accuracy & 0.5538/0.7201 & 0.6604/0.7348 & 0.6727/0.7307 & 0.7520 & 0.7666  \\
\hline
%\textcolor{blue}{Our Method} & 128k/32k & --- & 89.559 & 1024 TPUs & ---\\
%\hline
\end{tabular}
\label{table:resnet50_acc}
\end{table}

\subsection{Hyperparameters for scaling the batch size}
For $\bert$ and ImageNet training, we did not tune the hyperparameters of $\lamb$ optimizer when increasing the batch size. We use the square root LR scaling rule and linear-epoch warmup scheduling to automatically adjust learning rate. The details can be found in Tables \ref{table:hyper_parameters} and \ref{table:resnet_hyper_parameters}

\begin{table}[ht]
\renewcommand{\arraystretch}{1.3}
\caption{Untuned $\lamb$ for $\bert$ training across different batch sizes (fixed \#epochs). We use square root LR scaling and linear-epoch warmup. For example, batch size 32K needs to finish 15625 iterations. It uses 0.2$\times$15625 = 3125 iterations for learning rate warmup. $\bert$'s baseline achieved a F1 score of 90.395. We can achieve an even higher F1 score if we manually tune the hyperparameters.}

\centering
\begin{tabular}{|c|c|c|c|c|c|c|c|}
\hline
Batch Size & 512 & 1K & 2K & 4K & 8K & 16K & 32K\\
\hline
\hline
Learning Rate & $\frac{5}{2^{3.0}\times10^{3}}$ & $\frac{5}{2^{2.5}\times10^{3}}$ & $\frac{5}{2^{2.0}\times10^{3}}$ & $\frac{5}{2^{1.5}\times10^{3}}$ & $\frac{5}{2^{1.0}\times10^{3}}$ & $\frac{5}{2^{0.5}\times10^{3}}$ & $\frac{5}{2^{0.0}\times10^{3}}$\\
\hline
Warmup Ratio & $\frac{1}{320}$ & $\frac{1}{160}$ & $\frac{1}{80}$ & $\frac{1}{40}$ & $\frac{1}{20}$ & $\frac{1}{10}$ & $\frac{1}{5}$\\
\hline
F1 score & 91.752 & 91.761 & 91.946 & 91.137 & 91.263 & 91.345 & 91.475 \\
\hline
Exact Match & 85.090 & 85.260 & 85.355 & 84.172 & 84.901 & 84.816 & 84.939 \\
\hline
\end{tabular}
\label{table:hyper_parameters}
\end{table}

\begin{table}[ht]
\renewcommand{\arraystretch}{1.3}
\caption{Untuned $\lamb$ for ImageNet training with $\resnet$-50 for different batch sizes (90 epochs). We use square root LR scaling and linear-epoch warmup. The baseline \cite{goyal2017accurate} gets 76.3\% top-1 accuracy in 90 epochs. Stanford DAWN Bench \citep{coleman2017dawnbench} baseline achieves 93\% top-5 accuracy. $\lamb$ achieves both of them. $\lamb$ can achieve an even higher accuracy if we manually tune the hyperparameters.}
\centering
\begin{tabular}{|c|c|c|c|c|c|c|c|}
\hline
Batch Size & 512 & 1K & 2K & 4K & 8K & 16K & 32K\\
\hline
\hline
Learning Rate & $\frac{4}{2^{3.0}\times100}$ & $\frac{4}{2^{2.5}\times100}$ & $\frac{4}{2^{2.0}\times100}$ & $\frac{4}{2^{1.5}\times100}$ & $\frac{4}{2^{1.0}\times100}$ & $\frac{4}{2^{0.5}\times100}$ & $\frac{4}{2^{0.0}\times100}$\\
\hline
Warmup Epochs & 0.3125 & 0.625 & 1.25 & 2.5 & 5 & 10 & 20\\
\hline
Top-5 Accuracy & 0.9335 & 0.9349 & 0.9353 & 0.9332 & 0.9331 & 0.9322 & 0.9308 \\
\hline
Top-1 Accuracy & 0.7696 & 0.7706 & 0.7711 & 0.7692 & 0.7689 & 0.7666 & 0.7642 \\
\hline
%Exact Match & 85.090 & 85.260 & 85.355 & 84.172 & 84.901 & 84.816 & 84.939 \\
%\hline
\end{tabular}
\label{table:resnet_hyper_parameters}
\end{table}

%\vspace{-2mm}
\section{Conclusion}
%\vspace*{-3mm}
Large batch techniques are critical to speeding up deep neural network training. In this paper, we propose the $\lamb$ optimizer, which supports adaptive elementwise updating and layerwise learning rates. Furthermore, $\lamb$ is a general purpose optimizer that works for both small and large batches. We also provided theoretical analysis for the $\lamb$ optimizer, highlighting the cases where it performs better than standard $\sgd$. $\lamb$ achieves a better performance than existing optimizers for a wide range of applications.  By using $\lamb$, we are able to scale the batch size of $\bert$ pre-training to 64K without losing accuracy, thereby, reducing the $\bert$ training time from 3 days to around 76 minutes. $\lamb$ is also the first large batch adaptive solver that can achieve state-of-the-art accuracy on ImageNet training with $\resnet$-50.

\vspace{-2mm}
\section{Acknowledgement}
\vspace*{-3mm}
%This paper does not propose $\lars$ optimizer; we only provide its convergence analysis.
We want to thank the comments from George Dahl and Jeff Dean.
We want to thank Michael Banfield, Dehao Chen, Youlong Cheng, Sameer Kumar, and Zak Stone for TPU Pod support.
%This technical report is only for the research purpose.
%The TPU's speed in this paper should not be considered as Google's official number.
%The readers need to check Google's official documents for TPU's speed. 
%The content, views and conclusions presented in this paper do not necessarily reflect the position of Google.


\bibliography{iclr2020_conference}
\bibliographystyle{iclr2020_conference}

\appendix
%\section{Appendix}
%!TEX root = supp.tex
\renewcommand{\thesubsection}{\Alph{subsection}}
\newcommand\tab[1][5mm]{\hspace*{#1}}
\section*{Appendix}

\subsection{Additional Results}
\label{ap:more_results}
\paragraph{$k$-NN classification.}
\label{sec:knn}
In Tab.~\ref{tab:knn}, we evaluate the frozen representations given by ResNet-50 or ViT-small pre-trained with \OURS with two evaluation protocols: linear or $k$-NN.
For both evaluations, we extract representations from a pre-trained network without using any data augmentation.
Then, we perform classification either with weighted $k$-NN or with a linear regression learned with \texttt{cyanure} library~\cite{mairal2019cyanure}.
In Tab.~\ref{tab:knn} we see that ViT-S accuracies are better than accuracies obtained with RN50 both with a linear or a $k$-NN classifier.
However, the performance gap when using the $k$-NN evaluation is much more significant than when considering linear evaluation.
For example on ImageNet 1\%, ViT-S outperforms ResNet-50 by a large margin of $+14.1\%$ with $k$-NN evaluation.
This suggests that transformers architectures trained with \OURS might offer more model flexibility that benefits the $k$-NN evaluation.
$K$-NN classifiers have the great advantage of being fast and light to deploy, without requiring any domain adaptation.
Overall, ViT trained with \OURS provides features that combine particularly well with $k$-NN classifiers.

\begin{table}[h]
  \centering
\small
  \setlength{\tabcolsep}{5pt}
\caption{\textbf{$k$-NN and linear evaluation for ViT-S/16 and ResNet-50 pre-trained with \OURS.}
	We use ImageNet-1k~\cite{russakovsky2015imagenet} (``Inet''), Places205~\cite{zhou2014learning}, PASCAL VOC~\cite{everingham2010pascal} and Oxford-102 flowers (``FLOWERS'')~\cite{nilsback2008automated}.
ViT trained with \OURS provides features that are particularly $k$-NN friendly.}
  \begin{tabular}{@{}l ccg c ccg@{}}
    \toprule
    & \multicolumn{3}{c}{Logistic} && \multicolumn{3}{c}{$k$-NN} \\
    \cmidrule{2-4}
    \cmidrule{6-8}
	  & RN50 & ViT-S & $\Delta$ && RN50 & ViT-S & $\Delta$ \\
    \midrule
	  Inet   100\% & 72.1 & 75.7 & 3.6 && 67.5 & 74.5 & 7.0 \\
	  Inet   10\% & 67.8 & 72.2 & 4.4 && 59.3 & 69.1 & 9.8 \\
	  Inet    1\% & 55.1 & 64.5 & 9.4 && 47.2 & 61.3 & 14.1 \\
    Pl. 10\% & 53.4 & 52.1 & -1.3  && 46.9 & 48.6 & 1.7 \\
    Pl.  1\% & 46.5 & 46.3 & -0.2 && 39.2 & 41.3 & 2.1 \\
    VOC07     & 88.9 & 89.2 & 0.3 && 84.9 & 88.0 & 3.1 \\
    FLOWERS     & 95.6 & 96.4 & 0.8 && 87.9 & 89.1 & 1.2 \\
    %DOGS     & 75.8 & 84.0 & 8.2 && 62.1 & 79.2 & 17.1 \\
    %CUB     & 44.8 & 74.0 & 29.2 && 24.1 & 60.3 & 36.2 \\
    \midrule
    Average $\Delta$     & & & 2.4 && & & \bf 5.6 \\
    \bottomrule
  \end{tabular}
  \label{tab:knn}
\end{table}

\begin{table}[t]
  \centering
  \setlength{\tabcolsep}{.7em}
	\caption{
\textbf{ImageNet classification with different pretraining.}
Top-1 accuracy on ImageNet for supervised ViT-B/16 models using different pretrainings or using an additional pretrained convnet to guide the training.
The methods use different image resolution (``res.'') and training procedure (``tr. proc.''), i.e., data augmentation and optimization.
``MPP'' is \textit{Masked Patch Prediction}.
}
    \begin{tabular}{@{} lc c ccc @{}}
    \toprule
    \multicolumn{2}{c}{Pretraining} && & &  \\
    \cmidrule{1-2}
    method & data && res. & tr. proc. & Top-1\\
    \midrule
    \multicolumn{4}{@{}l}{\textit{Pretrain on additional data}}\\
	  MMP        &  JFT-300M &&384&\cite{dosovitskiy2020image}  & 79.9 \\
	  Supervised & JFT-300M  &&384&\cite{dosovitskiy2020image}  & 84.2 \\
    \midrule
    \multicolumn{4}{@{}l}{\textit{Train with additional model}}\\
	  Rand. init. & - &&224&\cite{touvron2020training}& 83.4 \\
    \midrule
    \multicolumn{4}{@{}l}{\textit{No additional data nor model}}\\
	  Rand. init. & -       &&224 &\cite{dosovitskiy2020image} & 77.9 \\
	  Rand. init. & -        &&224&\cite{touvron2020training}   & 81.8 \\
	  Supervised & ImNet &&224&\cite{touvron2020training}   & 81.9 \\
    \rowcolor{Light}
	  \OURS & ImNet      &&224&\cite{touvron2020training}   & 82.8 \\
	  %&&Ours & ViT-B/16 & \OURS & ImageNet & - \\
    \bottomrule
  \end{tabular}
  \label{tab:finetune}
\end{table}

\paragraph{Self-supervised ImageNet pretraining of ViT.}
In this experiment, we study the impact of pretraining a supervised ViT model with our method.
In Tab.~\ref{tab:finetune}, we compare the performance of supervised ViT models that are initialized with different pretraining or guided during training with an additional pretrained convnet.
The first set of models are pretrained with and without supervision on the large curated dataset composed of 300M images.
The second set of models are trained with hard knowledge distillation from a pretrained supervised RegNetY~\cite{radosavovic2020designing}.
The last set of models do not use any additional data nor models, and are initialized either randomly or after a pretraining with \OURS on ImageNet.
Compare to random initialization, pretraining with \OURS leads to a performance gain of +1\%.
This is not caused by a longer training since pretraining with supervision instead of \OURS does not improve performance.
Using self-supervised pretraining reduces the gap with models pretrained on extra data or distilled from a convnet.

\paragraph{Low-shot learning on ImageNet.}
We evaluate the features obtained with \OURS applied on ViT-S on low-shot learning.
In Tab.~\ref{tab:semisup}, we report the validation accuracy of a logistic regression trained on frozen features (\textsc{frozen}) with 1\% and 10\% labels.
The logistic regression is trained with the \texttt{cyanure} library~\cite{mairal2019cyanure}.
When comparing models with a similar number of parameters and image/sec, we observe that our features are on par with state-of-the-art semi-supervised models.
Interestingly, this performance is obtained by training a multi-class logistic regression on \emph{frozen features, without data augmentation nor finetuning}.
%When moving to larger models, we also observe that our features obtain competitive performance.

\begin{table}[t]
	\caption{
\textbf{Low-shot learning on ImageNet with frozen ViT features.}
We train a logistic regression on frozen features (\textsc{frozen}).
Note that this \textsc{frozen} evaluation is performed \emph{without any finetuning nor data augmentation}.
We report top-1 accuracy.
For reference, we show previously published results that uses finetuning and semi-supervised learning.}
    \centering
\small
    \begin{tabular}{@{}l l c c c@{}}
	    \toprule
        & && \multicolumn{2}{c}{Top 1}\\
        Method & Arch & Param. &  1\% &  10\% \\
\midrule
\multicolumn{5}{@{}l@{}}{\textit{Self-supervised pretraining with finetuning}}\\
        UDA~\cite{xie2020unsupervised} & RN50 & 23 & -- & 68.1 \\
        SimCLRv2~\cite{chen2020big} & RN50 & 23 & 57.9 & 68.4 \\
        BYOL~\cite{grill2020bootstrap} & RN50 & 23 & 53.2 & 68.8 \\
        SwAV~\cite{caron2020unsupervised} & RN50 & 23 & 53.9 & 70.2 \\
	SimCLRv2~\cite{chen2020exploring} & RN50w4 & 375 & 63.0 & 74.4\\
        BYOL~\cite{grill2020bootstrap} & RN200w2 & 250 & 71.2 & 77.7 \\
\midrule
\multicolumn{5}{@{}l@{}}{\textit{Semi-supervised methods}}\\
        SimCLRv2+KD~\cite{chen2020big} & RN50 & 23 & 60.0 & 70.5 \\
        SwAV+CT~\cite{assran2020recovering} & RN50 & 23 & -- & 70.8 \\
        FixMatch~\cite{sohn2020fixmatch} & RN50 & 23 & -- & 71.5 \\
        MPL~\cite{pham2020meta} & RN50 & 23 & -- & 73.9 \\
	SimCLRv2+KD~\cite{chen2020big} & RN152w3+SK & 794 & 76.6 & 80.9 \\
	\midrule
\multicolumn{5}{@{}l@{}}{\textit{Frozen self-supervised features}}\\
        \rowcolor{Light}
        \OURS\textsc{-frozen} & ViT-S/16 & 21 & 64.5 & 72.2 \\
	\bottomrule
    \end{tabular}
    \label{tab:semisup}
\end{table}

\begin{figure*}[t]
\centering
  \setlength{\tabcolsep}{3pt}
\begin{tabular}{ccc}
\includegraphics[width=0.35\linewidth]{678.pdf}&
%\includegraphics[width=0.4\linewidth]{4143.pdf}\\
\includegraphics[width=0.35\linewidth]{4833.pdf}&
\includegraphics[width=0.35\linewidth]{3296.pdf}\\
\end{tabular}
\caption{
	\textbf{Self-attention for a set of reference points.}
We visualize the self-attention module from the last block of a ViT-S/8 trained with \OURS.
The network is able to separate objects, though it has been trained with no supervision at all.
}
\label{fig:pointing}
\end{figure*}

\subsection{Methodology Comparison}
\label{ap:comp}
We compare the performance of different self-supervised frameworks, MoCo-v2~\cite{chen2020improved}, SwAV~\cite{caron2020unsupervised} and BYOL~\cite{grill2020bootstrap} when using convnet or ViT.
In Tab.~\ref{tab:vit_cnn}, we see that when trained with ResNet-50 (convnet), \OURS performs on par with SwAV and BYOL.
However, \OURS unravels its potential with ViT, outperforming MoCo-v2, SwAV and BYOL by large margins (+4.3\% with linear and +6.2\% with k-NN evaluations).
In the rest of this section, we perform ablations to better understand the performance of \OURS applied to ViT.
In particular, we provide a detailed comparison with methods that either use a momentum encoder, namely MoCo-v2 and BYOL, and methods that use multi-crop, namely SwAV.
\begin{table}[t]
\centering
  \caption{
	  \textbf{Methodology comparison for DEIT-small and ResNet-50.}
    We report ImageNet linear and $k$-NN evaluations validation accuracy after 300 epochs pre-training.
    All numbers are run by us and match or outperform published results.
  }
  \begin{tabular}{@{}ll cc c cc@{}}
    \toprule
  && \multicolumn{2}{c}{ResNet-50} && \multicolumn{2}{c}{ViT-small} \\
    \cmidrule{3-4}
    \cmidrule{6-7}
	  Method && Linear & $k$-NN && Linear & $k$-NN \\
    \midrule
	  MoCo-v2 && 71.1 & 62.9 && 71.6 & 62.0 \\
	  BYOL && 72.7 & 65.4 && 71.4 & 66.6 \\
	  SwAV && 74.1 & 65.4 && 71.8 & 64.7 \\
    \midrule
	  \OURS && \bf 74.5 & \bf 65.6  && \bf 76.1 & \bf 72.8 \\
    \bottomrule
  \end{tabular}
  \label{tab:vit_cnn}
\end{table}

\paragraph{Relation to MoCo-v2 and BYOL.}
In Tab.~\ref{tab:byol}, we present the impact of ablating components that differ between \OURS, MoCo-v2 and BYOL: the choice of loss, the predictor in the student head, the centering operation, the batch normalization in the projection heads, and finally, the multi-crop augmentation.
%BYOL shares many components with MoCo-v2, and hence we compare them jointly.
The loss in \OURS is a cross-entropy on sharpened softmax outputs (\texttt{CE}) while MoCo-v2 uses the InfoNCE contrastive loss (\texttt{INCE}) and BYOL a mean squared error on l2-normalized outputs (\texttt{MSE}).
No sharpening is applied with the \texttt{MSE} criterion.
%When using \texttt{MSE}, no sharpening is applied since we do not transform the output into probability distributions.
Though, \OURS surprisingly still works when changing the loss function to \texttt{MSE}, but this significantly alters the performance (see rows (\rownumber{1}, \rownumber{2}) and (\rownumber{4}, \rownumber{9})).
We also observe that adding a predictor has little impact (\rownumber{1}, \rownumber{3}).
However, in the case of BYOL, the predictor is critical to prevent collapse (\rownumber{7}, \rownumber{8}) which is consistent with previous studies~\cite{chen2020exploring,grill2020bootstrap}.
Interestingly, we observe that the teacher output centering avoids collapse without predictor nor batch normalizations in BYOL (\rownumber{7}, \rownumber{9}), though with a significant performance drop which can likely be explained by the fact that our centering operator is designed to work in combination with sharpening.
Finally, we observe that multi-crop works particularly well with \OURS and MoCo-v2, removing it hurts performance by $2-4\%$ (\rownumber{1} versus \rownumber{4} and, \rownumber{5} versus \rownumber{6}).
Adding multi-crop to BYOL does not work out-of-the-box (\rownumber{7}, \rownumber{10}) as detailed in Appendix~\ref{ap:mc} and further adaptation may be required.
%Our main conclusions are that MoCo-v2 works very well with multi-crop on ViT while any change made to BYOL results in a drop of performance.

\begin{table}[t]
  \centering
\small
  \setlength{\tabcolsep}{4.3pt}
	\caption{
    \textbf{Relation to MoCo-v2 and BYOL.}
	We ablate the components that differ between \OURS, MoCo-v2 and BYOL: the loss function (cross-entropy, \texttt{CE}, versus InfoNCE, \texttt{INCE}, versus mean-square error, \texttt{MSE}), the multi-crop training, the centering operator, the batch normalization in the projection heads and the student predictor. Models are run for 300 epochs with ViT-S/16. We report top-1 accuracy on ImageNet linear evaluation.
  }
  \begin{tabular}{@{}llcccccc@{}}
    \toprule
	  & Method & Loss & multi-crop & Center. & BN & Pred. & Top-1 \\
    \midrule
\rownumber{1}&	  \OURS & \texttt{CE} & \checkmark & \checkmark & & & 76.1 \\
\rownumber{2}&  -- & \texttt{MSE} & \checkmark & \checkmark & & & 62.4 \\
\rownumber{3}&  -- & \texttt{CE} & \checkmark & \checkmark & & \checkmark & 75.6 \\
\rownumber{4}&  -- & \texttt{CE} & & \checkmark & & & 72.5 \\
\arrayrulecolor{black!20}\midrule
\rownumber{5}&  MoCov2 & \texttt{INCE} & & & \checkmark &  & 71.4 \\
\rownumber{6}&   & \texttt{INCE} & \checkmark & & \checkmark &  & 73.4 \\
\arrayrulecolor{black!20}\midrule
\rownumber{7}&  BYOL & \texttt{MSE} & & & \checkmark & \checkmark & 71.4 \\
\rownumber{8}&  -- & \texttt{MSE} & & & \checkmark &  & \pzo0.1 \\
\rownumber{9}&  -- & \texttt{MSE} & & \checkmark &  &  & 52.6 \\
\rownumber{10}&  -- & \texttt{MSE} & \checkmark  & & \checkmark & \checkmark & 64.8 \\
   \arrayrulecolor{black} \bottomrule
  \end{tabular}
\label{tab:byol}
\end{table}


\begin{table}[t]
  \centering
  %\setlength{\tabcolsep}{2.5pt}
	\caption{\textbf{Relation to SwAV.}
We vary the operation on the teacher output between centering, a softmax applied over the batch dimension and the Sinkhorn-Knopp algorithm. 
We also ablate the Momentum encoder by replacing it with a hard copy of the student with a stop-gradient as in SwAV.
Models are run for 300 epochs with ViT-S/16. We report top-1 accuracy on ImageNet linear evaluation.
	}
  \begin{tabular}{@{}llccc@{}}
    \toprule
	 & Method & Momentum & Operation & Top-1 \\
    \midrule
\rownumber{1}&  \OURS & \checkmark & \texttt{Centering} & 76.1 \\
\rownumber{2}&  -- & \checkmark & \texttt{Softmax(batch)} & 75.8 \\
\rownumber{3}&  -- & \checkmark & \texttt{Sinkhorn-Knopp} & 76.0 \\
\rownumber{4}&  -- & & \texttt{Centering} & \pzo0.1 \\
\rownumber{5}&  -- & & \texttt{Softmax(batch)} & 72.2 \\
\rownumber{6}&  SwAV & & \texttt{Sinkhorn-Knopp} & 71.8 \\
    \bottomrule
  \end{tabular}
  \label{tab:swav}
\end{table}

\paragraph{Relation to SwAV.}
In Tab.~\ref{tab:swav}, we evaluate the differences between \OURS and SwAV: the presence of the momentum encoder and the operation on top of the teacher output.
In absence of the momentum, a copy of the student with a stop-gradient is used.
We consider three operations on the teacher output: \texttt{Centering}, \texttt{Sinkhorn-Knopp} or a \texttt{Softmax} along the batch axis.
The \texttt{Softmax} is similar to a single Sinkhorn-Knopp iteration as detailed in the next paragraph.
First, these ablations show that using a momentum encoder significantly improves the performance for ViT (\rownumber{3} versus \rownumber{6}, and \rownumber{2} versus \rownumber{5}).
Second, the momentum encoder also avoids collapse when using only centering (row \rownumber{1}).
In the absence of momentum, centering the outputs does not work (\rownumber{4}) and more advanced operations are required (\rownumber{5}, \rownumber{6}).
Overall, these ablations highlight the importance of the momentum encoder, not only for performance but also to stabilize training, removing the need for normalization beyond centering.

\paragraph{Details on the \texttt{Softmax(batch)} variant.}
The iterative Sinkhorn-Knopp algorithm~\cite{cuturi2013sinkhorn} used in SwAV~\cite{caron2020unsupervised} is implemented simply with the following PyTorch style code.
\begin{python}
# x is n-by-K
# tau is Sinkhorn regularization param
x = exp(x / tau)
for _ in range(num_iters): # 1 iter of Sinkhorn
	# total weight per dimension (or cluster)
	c = sum(x, dim=0, keepdim=True) 
	x /= c

	# total weight per sample
	n = sum(x, dim=1, keepdim=True) 
	# x sums to 1 for each sample (assignment)
	x /= n 
\end{python}
When performing a single Sinkhorn iteration (\texttt{num\_iters=1}) the implementation can be highly simplified into only two lines of code, which is our \texttt{softmax(batch)} variant:
\begin{python}
x = softmax(x / tau, dim=0)
x /= sum(x, dim=1, keepdim=True) 
\end{python}
We have seen in Tab.~\ref{tab:swav} that this highly simplified variant of SwAV works competitively with SwAV.
Intuitively, the \texttt{softmax} operation on the batch axis allows to select for each dimension (or ``cluster'') its best matches in the batch.


\paragraph{Validating our implementation.}
We observe in Tab.~\ref{tab:vit_cnn} that our reproduction of BYOL, MoCo-v2, SwAV matches or outperforms the corresponding published numbers with ResNet-50.
Indeed, we obtain $72.7\%$ for BYOL while \cite{grill2020bootstrap} report $72.5\%$ in this $300$-epochs setting.
We obtain $71.1\%$ for MoCo after $300$ epochs of training while \cite{chen2020improved} report $71.1\%$ after $800$ epochs of training.
Our improvement compared to the implementation of \cite{chen2020improved} can be explained by the use of a larger projection head (3-layer, use of batch-normalizations and projection dimension of $256$).
%We reproduce the published number for SwAV after $400$-epochs training: $74.6\%$.
%We obtain $74.1\%$ for SwAV after $300$ epochs, which is a setting not reported in \cite{caron2020unsupervised}.

\paragraph{Relation to other works.}
\OURS is also related to UIC~\cite{chen2020unsupervised} that use outputs from the previous epoch as hard pseudo-labels for ``unsupervised classification''.
However, we use centering to prevent collapse while UIC resorts to balance sampling techniques as in~\cite{caron2018deep}.
Our work can be interpreted as a soft UIC variant with momentum teacher.

The concurrent work CsMI~\cite{xu2021seed} also exhibits strong performance with simple k-NN classifiers on ImageNet, even with convnets.
As \OURS, CsMI combines a momentum network and multi-crop training, which we have seen are both crucial for good k-NN performance in our experiments with ViTs.
%Interestingly, unlike our work, CsMI reports excellent $k$-NN performance even with a ResNet-50 (72.4\%).
We believe studying this work would help us identifying more precisely the components important for good $k$-NN performance and leave this investigation for future work.

\subsection{Projection Head}
\label{ap:projhead}
Similarly to other self-supervised frameworks, using a projection head~\cite{chen2020simple} improves greatly the accuracy of our method.
The projection head starts with a $n$-layer multi-layer perceptron (MLP).
The hidden layers are 2048d and are with gaussian error linear units (GELU) activations.
The last layer of the MLP is without GELU.
Then we apply a $\ell_2$ normalization and a weight normalized fully connected layer~\cite{chen2020exploring,salimans2016weight} with $K$ dimensions.
This design is inspired from the projection head with a ``prototype layer'' used in SwAV~\cite{caron2020unsupervised}.
We do not apply batch normalizations.
%In the following, we study the effect of the l2-normalization bottleneck, the number of linear layers in the projection head, the output dimension $K$, the choice of activation unit and the impact of adding batch normalizations.

\paragraph{BN-free system.}
Unlike standard convnets, ViT architectures do not use batch normalizations (BN) by default.
\begin{table}[h!]
\vspace{-0.8em}
  \centering
  \begin{tabular}{@{}l c c@{}}
	  ViT-S, 100 epochs & heads w/o BN & heads w/ BN \\
    \midrule
	  $k$-NN top-1 & \colorbox{Light}{69.7} & 68.6 \\
  \end{tabular}
\vspace{-0.8em}
\end{table}
Therefore, when applying \OURS to ViT we do not use any BN also in the projection heads.
In this table we evaluate the impact of adding BN in the heads.
We observe that adding BN in the projection heads has little impact, showing that BN is not important in our framework.
\emph{Overall, when applying \OURS to ViT, we do not use any BN anywhere, making the system entirely BN-free.}
This is a great advantage of \OURS + ViT to work at state-of-the-art performance without requiring any BN.
Indeed, training with BN typically slows down trainings considerably, especially when these BN modules need to be synchronized across processes~\cite{he2020momentum,caron2020unsupervised,caron2019unsupervised,grill2020bootstrap}.
\begin{figure}[h!]
\centering
\includegraphics[width=\linewidth]{head_design.pdf}
	\caption{
    \textbf{Projection head design w/ or w/o l2-norm bottleneck.}
  }
\label{fig:proj}
\end{figure}

\paragraph{L2-normalization bottleneck in projection head.}
We illustrate the design of the projection head with or without l2-normalization bottleneck in Fig.~\ref{fig:proj}.
\begin{table}[h!]
\vspace{-0.8em}
  \centering
  \begin{tabular}{@{}l c c c c@{}}
	  \# proj. head linear layers & $1$ & $2$ & $3$ & $4$\\
    \midrule
	  w/ l2-norm bottleneck & -- & 62.2 & 68.0 & \colorbox{Light}{69.3} \\
	  w/o l2-norm bottleneck & 61.6 & 62.9 & 0.1 & 0.1 \\
  \end{tabular}
\vspace{-0.8em}
\end{table}
We evaluate the accuracy of \OURS models trained with or without l2-normalization bottleneck and we vary the number of linear layers in the projection head.
With l2 bottleneck, the total number of linear layers is $n + 1$ ($n$ from the MLP and $1$ from the weight normalized layer) while without bottleneck the total number of linear layers is $n$ in the head.
In this table, we report ImageNet top-1 $k$-NN evaluation accuracy after 100 epochs pre-training with ViT-S/16.
The output dimensionality $K$ is set to $4096$ in this experiment.
We observe that \OURS training fails without the l2-normalization bottleneck when increasing the depth of the projection head.
L2-normalization bottleneck stabilizes the training of \OURS with deep projection head.
We observe that increasing the depth of the projection head improves accuracy.
Our default is to use a total of 4 linear layers: 3 are in the MLP and one is after the l2 bottleneck.

\paragraph{Output dimension.}
In this table, we evaluate the effect of varying the output dimensionality $K$.
\begin{table}[h!]
\vspace{-0.8em}
  \centering
  \begin{tabular}{@{}l c c c c c@{}}
	  $K$ & 1024 & 4096 & 16384 & 65536 & 262144 \\
    \midrule
	  $k$-NN top-1 & 67.8 & 69.3 & 69.2 & \colorbox{Light}{69.7} & 69.1 \\
  \end{tabular}
\vspace{-0.8em}
\end{table}
We observe that a large output dimensionality improves the performance.
We note that the use of l2-normalization bottleneck permits to use a large output dimension with a moderate increase in the total number of parameters.
Our default is to use $K$ equals to 65536 and $d=256$ for the bottleneck.

\paragraph{GELU activations.}
By default, the activations used in ViT are gaussian error linear units (GELU).
\begin{table}[h!]
\vspace{-0.8em}
  \centering
  \begin{tabular}{@{}l c c@{}}
	  ViT-S, 100 epochs & heads w/ GELU & heads w/ ReLU \\
    \midrule
	  $k$-NN top-1 & \colorbox{Light}{69.7} & 68.9 \\
  \end{tabular}
\vspace{-0.8em}
\end{table}
Therefore, for consistency within the architecture, we choose to use GELU also in the projection head.
We evaluate the effect of using ReLU instead of GELU in this table and observe that changing the activation unit to ReLU has relatively little impact.
%It is possible that re-tuning the hyperparameters would allow to recover the $0.8\%$ performance gap between our default and the heads with ReLU.


\subsection{Additional Ablations}
\label{ap:ablations}
We have detailed in the main paper that the combination of centering and sharpening is important to avoid collapse in \OURS.
We ablate the hyperparameters for these two operations in the following.
We also study the impact of training length and some design choices for the ViT networks.

\paragraph{Online centering.}
We study the impact of the smoothing parameters in the update rule for the center $c$ used in the output of the teacher network.
\begin{table}[h!]
\vspace{-0.8em}
  \centering
  \begin{tabular}{@{}l c c c c@{}}
	  $m$ & 0 & 0.9 & 0.99 & 0.999 \\
    \midrule
	  $k$-NN top-1 & 69.1 & \colorbox{Light}{69.7} & 69.4 & 0.1 \\
  \end{tabular}
\vspace{-0.8em}
\end{table}
The convergence is robust to a wide range of smoothing, and the model only collapses when the update is too slow, i.e., $m = 0.999$.

\paragraph{Sharpening.}
We enforce sharp targets by tuning the teacher softmax temperature parameter $\tau_t$.
In this table, we observe that a temperature lower than $0.06$ is required to avoid collapse.
\begin{table}[h!]
\vspace{-0.8em}
  \centering
  \setlength{\tabcolsep}{4pt}
  \begin{tabular}{@{}l c c c c c c@{}}
	  $\tau_t$ & $0$ & $0.02$ & $0.04$ & $0.06$ & $0.08$ & $0.04 \rightarrow 0.07$ \\
    \midrule
	  $k$-NN top-1 & 43.9 & 66.7 & 69.6 & 68.7 & 0.1 & \colorbox{Light}{69.7} \\
  \end{tabular}
\vspace{-0.8em}
\end{table}
When the temperature is higher than $0.06$, the training loss consistently converges to $ln(K)$.
However, we have observed that using higher temperature than $0.06$ does not collapse if we start the training from a smaller value and increase it during the first epochs.
%\OURS is particularly sensitive to collapse at the beginning of training.
%Our experiments suggest that we should seek for the maximum temperature that does not collapse.
In practice, we use a linear warm-up for $\tau_t$ from $0.04$ to $0.07$ during the first $30$ epochs of training.
Finally, note that $\tau \rightarrow 0$ (extreme sharpening) correspond to the \texttt{argmax} operation and leads to one-hot hard distributions.

\paragraph{Longer training.}
We observe in this table that longer training improves the performance of \OURS applied to ViT-Small.
\vspace{-0.8em}
\begin{table}[h!]
  \centering
\begin{tabular}{@{}l c c c@{}}
	\OURS ViT-S & 100-ep & 300-ep & 800-ep \\
\midrule
	$k$-NN top-1 & 70.9 & 72.8 & \colorbox{Light}{74.5} \\
\end{tabular}
\vspace{-0.8em}
\end{table}
This observation is consistent with self-supervised results obtained with convolutional architectures~\cite{chen2020simple}.
We note that in our experiments with BYOL on ViT-S, training longer than $300$ epochs has been leading to worse performance compare our $300$ epochs run.
For this reason we report BYOL for 300 epochs in Tab.~\ref{tab:sota} while SwAV, MoCo-v2 and DINO are trained for 800 epochs.
%Results in Tab. 1 of the main paper are with $800$ epochs training for ViT-S but $300$ epochs only for the other considered models (ViT-S/8, ViT-B/16, ViT-B/8) to save computational budget.
%We have not explored longer trainings for these models yet.

\paragraph{The teacher outperforms the student.}
We have shown in Fig.~\ref{fig:mom} that the momentum teacher outperforms the student with ViT and we show in this Figure that it is also the case with ResNet-50.
\begin{figure}[h!]
\vspace{-0.8em}
\centering
\includegraphics[width=0.48\linewidth]{figure_mom_rn50.pdf}
\vspace{-0.8em}
\end{figure}
The fact that the teacher continually outperforms the student further encourages the interpretation of \OURS as a form of Mean Teacher~\cite{tarvainen2017mean} self-distillation.
Indeed, as motivated in Tarvainen et al.~\cite{tarvainen2017mean}, weight averaging usually produces a better model than the individual models from each iteration~\cite{polyak1992acceleration}.
By aiming a target obtained with a teacher better than the student, the student's representations improve.
Consequently, the teacher also improves since it is built directly from the student weights.

\paragraph{Self-attention maps from supervised versus self-supervised learning.}
We evaluate the masks obtained by thresholding the self-attention maps to keep 80\% of the mass.
\begin{table}[h!]
\vspace{-0.8em}
  \centering
  \begin{tabular}{@{}l c@{}}
	  \toprule
	  ViT-S/16 weights & \\
	  \midrule
	  Random weights & 22.0 \\
	  Supervised & 27.3 \\
	  \midrule
	  \OURS & 45.9 \\
	  \OURS w/o multicrop & 45.1 \\
	  MoCo-v2 & 46.3 \\
	  BYOL & 47.8 \\
	  SwAV & 46.8 \\
	\bottomrule
  \end{tabular}
\vspace{-0.8em}
\end{table}
We compare the Jaccard similarity between the ground truth and these masks on the validation images of PASCAL VOC12 dataset for different ViT-S trained with different frameworks.
The properties that self-attention maps from ViT explicitly contain the scene layout and, in particular, object boundaries is observed across different self-supervised methods.

\paragraph{Impact of the number of heads in ViT-S.}
We study the impact of the number of heads in ViT-S on the accuracy and throughput (images processed per second at inference time on a singe V100 GPU).
\begin{table}[h!]
\vspace{-0.8em}
  \centering
  \begin{tabular}{@{}l c c c c c@{}}
	  \# heads & dim & dim/head & \# params & im/sec & $k$-NN \\
    \midrule
    \rowcolor{Light}
	  6 & 384 & 64 & 21 & 1007 & 72.8 \\
	  8 & 384 & 48 & 21 & 971 & 73.1 \\
	  12 & 384 & 32 & 21 & 927 & 73.7 \\
	  16 & 384 & 24 & 21 & 860 & 73.8 \\
  \end{tabular}
\vspace{-0.8em}
\end{table}
We find that increasing the number of heads improves the performance, at the cost of a slighlty worse throughput.
In our paper, all experiments are run with the default model DeiT-S~\cite{touvron2020training}, i.e. with $6$ heads only.

\subsection{Multi-crop}
\label{ap:mc}
In this Appendix, we study a core component of \OURS: multi-crop training~\cite{caron2020unsupervised}.


\paragraph{Range of scales in multi-crop.}
For generating the different views, we use the \texttt{RandomResizedCrop} method from \texttt{torchvision.transforms} module in PyTorch.
\begin{table}[h!]
\vspace{-0.8em}
\centering
  \begin{tabular}{@{}l c c c c c@{}}
	  (0.05, $s$), ($s$, 1), $s$: & 0.08 & 0.16 & 0.24 & 0.32 & 0.48 \\
    \midrule
	  $k$-NN top-1 & 65.6 & 68.0 & 69.7 & 69.8 & 69.5 \\
  \end{tabular}
\vspace{-0.8em}
\end{table}
We sample two global views with scale range $(s, 1)$ before resizing them to $224^2$ and $6$ local views with scale sampled in the range $(0.05, s)$ resized to $96^2$ pixels.
Note that we arbitrarily choose to have non-overlapping scaling range for the global and local views following the original design of SwAV.
However, the ranges could definitely be overlapping and experimenting with finer hyperparameters search could lead to a more optimal setting.
In this table, we vary the parameter $s$ that controls the range of scales used in multi-crop and find the optimum to be around $0.3$ in our experiments.
We note that this is higher than the parameter used in SwAV which is of $0.14$.

\paragraph{Multi-crop in different self-supervised frameworks.}
We compare different recent self-supervised learning frameworks, namely MoCo-v2~\cite{chen2020improved}, BYOL~\cite{grill2020bootstrap} and SwAV~\cite{caron2020unsupervised} with ViT-S/16 architecture.
\begin{table}[h!]
\vspace{-0.8em}
  \centering
\begin{tabular}{@{}l cc c cc@{}}
\toprule
crops	& \multicolumn{2}{c}{$2 \times 224^2$} && \multicolumn{2}{c}{$2 \times 224^2 + 6 \times 96^2$} \\
    \cmidrule{2-3}
    \cmidrule{5-6}
	eval & $k$-NN & linear && $k$-NN & linear  \\
\midrule
	BYOL & 66.6 & 71.4 && 59.8 & 64.8 \\
	SwAV & 60.5 & 68.5 && 64.7 & 71.8 \\
	MoCo-v2 & 62.0 & 71.6 && 65.4 & 73.4 \\
        \rowcolor{Light}
	\OURS & \bf 67.9 & \bf 72.5 && \bf 72.7 & \bf 75.9 \\
\bottomrule
\vspace{-0.8em}
\end{tabular}
\end{table}
For fair comparisons, all models are pretrained either with two $224^2$ crops or with multi-crop~\cite{caron2020unsupervised} training, i.e. two $224^2$ crops and six $96^2$ crops for each image.
We report $k$-NN and linear probing evaluations after 300 epochs of training.
Multi-crop does not benefit all frameworks equally, which has been ignored in benchmarks considering only the two crops setting~\cite{chen2020exploring}.
The effectiveness of multi-crop depends on the considered framework, which positions multi-crop as a core component of a model and not a simple ``add-ons'' that will boost any framework the same way.
Without multi-crop, \OURS has better accuracy than other frameworks, though by a moderate margin (1\%).
Remarkably, \OURS benefits the most from multi-crop training ($+3.4\%$ in linear eval).
Interestingly, we also observe that the ranking of the frameworks depends on the evaluation protocol considered.

\paragraph{Training BYOL with multi-crop.}
When applying multi-crop to BYOL with ViT-S, we observe the transfer performance is higher than the baseline without multi-crop for the first training epochs.
\begin{figure}[h]
\vspace{-0.8em}
\centering
\includegraphics[width=0.5\linewidth]{figure_byol.pdf}
\vspace{-0.8em}
\end{figure}
However, the transfer performance growth rate is slowing down and declines after a certain amount of training.
We have performed learning rate, weight decay, multi-crop parameters sweeps for this setting and systematically observe the same pattern.
More precisely, we experiment with \{$1e^{-5}$, $3e^{-5}$, $1e^{-4}$, $3e^{-4}$, $1e^{-3}$, $3e^{-3}$\} for learning rate base values, with \{$0.02$, $0.05$, $0.1$\} for weight decay and with different number of small crops: \{2, 4, 6\}.
All our runs are performed with synchronized batch normalizations in the heads.
When using a low learning rate, we did not observe the performance break point, i.e. the transfer performance was improving continually during training, but the overall accuracy was low.
We have tried a run with multi-crop training on ResNet-50 where we also observe the same behavior.
Since integrating multi-crop training to BYOL is not the focus of this study we did not push that direction further.
However, we believe this is worth investigating why multi-crop does not combine well with BYOL in our experiments and leave this for future work.

\subsection{Evaluation Protocols}
\subsubsection{$k$-NN classification}
Following the setting of Wu~\etal~\cite{wu2018unsupervised}, we evaluate the quality of features with a simple weighted $k$ Nearest Neighbor classifier.
We freeze the pretrained model to compute and store the features of the training data of the downstream task.
To classify a test image $x$, we compute its representation and compare it against all stored training features $T$.
The representation of an image is given by the output \texttt{[CLS]} token: it has dimensionality $d=384$ for ViT-S and $d=768$ for ViT-B.
The top $k$ NN (denoted $\mathcal{N}_k$) are used to make a prediction via weighted voting.
Specifically, the class $c$ gets a total weight of $\sum_{i \in \mathcal{N}_k} \alpha_i \mathbf{1}_{c_i = c}$, where $\alpha_i$ is a contribution weight.
We use $\alpha_i = \exp(T_i x / \tau)$ with $\tau$ equals to $0.07$ as in~\cite{wu2018unsupervised} which we do not tune.
We evaluate different values for $k$ and find that $k=20$ is consistently leading to the best accuracy across our runs.
This evaluation protocol does not require hyperparameter tuning, nor data augmentation and can be run with only one pass over the downstream dataset.

\subsubsection{Linear classification}
Following common practice in self-supervised learning, we evaluate the representation quality with a linear classifier.
The projection head is removed, and we train a supervised linear classifier on top of frozen features.
This linear classifier is trained with SGD and a batch size of $1024$ during $100$ epochs on ImageNet.
We do not apply weight decay.
For each model, we sweep the learning rate value.
During training, we apply only random resizes crops (with default parameters from PyTorch \texttt{RandomResizedCrop}) and horizontal flips as data augmentation.
We report central-crop top-1 accuracy.
When evaluating convnets, the common practice is to perform global average pooling on the final feature map before the linear classifier.
In the following, we describe how we adapt this design when evaluating ViTs.

\paragraph{ViT-S representations for linear eval.}
Following the \emph{feature-based} evaluations in BERT~\cite{devlin2018bert}, we concatenate the \texttt{[CLS]} tokens from the $l$ last layers.
\begin{table}[h!]
\vspace{-0.8em}
\small
  \centering
  \begin{tabular}{@{}l c c c c@{}}
	  concatenate $l$ last layers & $1$ & $2$ & $4$ & $6$\\
    \midrule
	  representation dim & 384 & 768 & 1536 & 2304 \\
	  ViT-S/16 linear eval & 76.1 & 76.6 & \colorbox{Light}{77.0} & 77.0 \\
  \end{tabular}
\vspace{-0.8em}
\end{table}
We experiment with the concatenation of a different number $l$ of layers and similarly to~\cite{devlin2018bert} we find $l=4$ to be optimal.
%We use the same setting when evaluating DeiT-S/8.

\paragraph{ViT-B representations for linear eval.}
With ViT-B we did not find that concatenating the representations from the last $l$ layers to provide any performance gain, and consider the final layer only ($l=1$).
\begin{table}[h!]
\small
\vspace{-0.8em}
  \centering
  \setlength{\tabcolsep}{4pt}
  \begin{tabular}{@{}l c c@{}}
	  pooling strategy & \texttt{[CLS]} tok. & concatenate \texttt{[CLS]} tok.\\
	  & only &  and avgpooled patch tok. \\
    \midrule
	  representation dim & 768 & 1536 \\
	  ViT-B/16 linear eval & 78.0 & \colorbox{Light}{78.2} \\
  \end{tabular}
\vspace{-0.8em}
\end{table}
In this setting, we adapt the pipeline used in convnets with global average pooling on the output patch tokens.
We concatenate these pooled features to the final \texttt{[CLS]} output token.
%We use the same setting when evaluating ViT-B/8.

\subsection{Self-Attention Visualizations}
\label{ap:visu}
We provide more self-attention visualizations in Fig.~\ref{fig:pointing} and in Fig.~\ref{fig:all}.
The images are randomly selected from COCO validation set, and are not used during training of \OURS.
In Fig.~\ref{fig:pointing}, we show the self-attention from the last layer of a \OURS ViT-S/8 for several reference points.

\subsection{Class Representation}
As a final visualization, we propose to look at the distribution of ImageNet concepts in the feature space from \OURS.
We represent each ImageNet class with the average feature vector for its validation images.
We reduce the dimension of these features to 30 with PCA, and run t-SNE with a perplexity of 20, a learning rate of 200 for 5000 iterations.
We present the resulting class embeddings in Fig.~\ref{fig:tsne}.
Our model recovers structures between classes: similar animal species are grouped together, forming coherent clusters of birds (top) or dogs, and especially terriers (far right).

\begin{figure*}
\centering
\setlength{\tabcolsep}{0.5pt}
\begin{tabular}{c cc ccc cc ccc cc c cc ccc cc ccc}
	&&& \multicolumn{3}{c}{\OURS} &&& \multicolumn{3}{c}{Supervised} &&& &&& \multicolumn{3}{c}{\OURS} &&& \multicolumn{3}{c}{Supervised} \\
    \cmidrule{4-6}
    \cmidrule{9-11}
    \cmidrule{17-19}
    \cmidrule{22-24}
\includegraphics[width=0.07\linewidth]{3478.png} &&&
\includegraphics[width=0.07\linewidth]{3478dino-depth12-head2.png} &
\includegraphics[width=0.07\linewidth]{3478dino-depth12-head0.png} &
\includegraphics[width=0.07\linewidth]{3478dino-depth12-head5.png} &&&
\includegraphics[width=0.07\linewidth]{3478sup-depth12-head0.png} &
\includegraphics[width=0.07\linewidth]{3478sup-depth12-head1.png} &
\includegraphics[width=0.07\linewidth]{3478sup-depth12-head2.png}
&&&
\includegraphics[width=0.07\linewidth]{944.png} &&&
\includegraphics[width=0.07\linewidth]{944dino-depth12-head0.png} &
\includegraphics[width=0.07\linewidth]{944dino-depth12-head2.png} &
\includegraphics[width=0.07\linewidth]{944dino-depth12-head3.png} &&&
\includegraphics[width=0.07\linewidth]{944sup-depth12-head0.png} &
\includegraphics[width=0.07\linewidth]{944sup-depth12-head1.png} &
\includegraphics[width=0.07\linewidth]{944sup-depth12-head2.png}
\\
\includegraphics[width=0.07\linewidth]{4323.png} &&&
\includegraphics[width=0.07\linewidth]{4323dino-depth12-head1.png} &
\includegraphics[width=0.07\linewidth]{4323dino-depth12-head2.png} &
\includegraphics[width=0.07\linewidth]{4323dino-depth12-head5.png} &&&
\includegraphics[width=0.07\linewidth]{4323sup-depth12-head0.png} &
\includegraphics[width=0.07\linewidth]{4323sup-depth12-head1.png} &
\includegraphics[width=0.07\linewidth]{4323sup-depth12-head2.png}
%\\
&&&
\includegraphics[width=0.07\linewidth]{4389.png} &&&
\includegraphics[width=0.07\linewidth]{4389dino-depth12-head2.png} &
\includegraphics[width=0.07\linewidth]{4389dino-depth12-head4.png} &
\includegraphics[width=0.07\linewidth]{4389dino-depth12-head5.png} &&&
\includegraphics[width=0.07\linewidth]{4389sup-depth12-head0.png} &
\includegraphics[width=0.07\linewidth]{4389sup-depth12-head1.png} &
\includegraphics[width=0.07\linewidth]{4389sup-depth12-head2.png}
\\
\includegraphics[width=0.07\linewidth]{102.png} &&&
\includegraphics[width=0.07\linewidth]{102dino-depth12-head2.png} &
\includegraphics[width=0.07\linewidth]{102dino-depth12-head4.png} &
\includegraphics[width=0.07\linewidth]{102dino-depth12-head5.png} &&&
\includegraphics[width=0.07\linewidth]{102sup-depth12-head0.png} &
\includegraphics[width=0.07\linewidth]{102sup-depth12-head1.png} &
\includegraphics[width=0.07\linewidth]{102sup-depth12-head2.png}
%\\
&&&
\includegraphics[width=0.07\linewidth]{4870.png} &&&
\includegraphics[width=0.07\linewidth]{4870dino-depth12-head1.png} &
\includegraphics[width=0.07\linewidth]{4870dino-depth12-head3.png} &
\includegraphics[width=0.07\linewidth]{4870dino-depth12-head2.png} &&&
\includegraphics[width=0.07\linewidth]{4870sup-depth12-head0.png} &
\includegraphics[width=0.07\linewidth]{4870sup-depth12-head1.png} &
\includegraphics[width=0.07\linewidth]{4870sup-depth12-head2.png}
\\
\includegraphics[width=0.07\linewidth]{135.png} &&&
\includegraphics[width=0.07\linewidth]{135dino-depth12-head3.png} &
\includegraphics[width=0.07\linewidth]{135dino-depth12-head1.png} &
\includegraphics[width=0.07\linewidth]{135dino-depth12-head4.png} &&&
\includegraphics[width=0.07\linewidth]{135sup-depth12-head0.png} &
\includegraphics[width=0.07\linewidth]{135sup-depth12-head1.png} &
\includegraphics[width=0.07\linewidth]{135sup-depth12-head2.png}
%\\
&&&
\includegraphics[width=0.07\linewidth]{2559.png} &&&
\includegraphics[width=0.07\linewidth]{2559dino-depth12-head0.png} &
\includegraphics[width=0.07\linewidth]{2559dino-depth12-head2.png} &
\includegraphics[width=0.07\linewidth]{2559dino-depth12-head1.png} &&&
\includegraphics[width=0.07\linewidth]{2559sup-depth12-head0.png} &
\includegraphics[width=0.07\linewidth]{2559sup-depth12-head1.png} &
\includegraphics[width=0.07\linewidth]{2559sup-depth12-head2.png}
\\
\includegraphics[width=0.07\linewidth]{4466.png} &&&
\includegraphics[width=0.07\linewidth]{4466dino-depth12-head2.png} &
\includegraphics[width=0.07\linewidth]{4466dino-depth12-head0.png} &
\includegraphics[width=0.07\linewidth]{4466dino-depth12-head5.png} &&&
\includegraphics[width=0.07\linewidth]{4466sup-depth12-head0.png} &
\includegraphics[width=0.07\linewidth]{4466sup-depth12-head1.png} &
\includegraphics[width=0.07\linewidth]{4466sup-depth12-head2.png}
%\\
&&&
\includegraphics[width=0.07\linewidth]{3689.png} &&&
\includegraphics[width=0.07\linewidth]{3689dino-depth12-head0.png} &
\includegraphics[width=0.07\linewidth]{3689dino-depth12-head4.png} &
\includegraphics[width=0.07\linewidth]{3689dino-depth12-head2.png} &&&
\includegraphics[width=0.07\linewidth]{3689sup-depth12-head0.png} &
\includegraphics[width=0.07\linewidth]{3689sup-depth12-head1.png} &
\includegraphics[width=0.07\linewidth]{3689sup-depth12-head2.png}
\\
\includegraphics[width=0.07\linewidth]{2076.png} &&&
\includegraphics[width=0.07\linewidth]{2076dino-depth12-head1.png} &
\includegraphics[width=0.07\linewidth]{2076dino-depth12-head0.png} &
\includegraphics[width=0.07\linewidth]{2076dino-depth12-head4.png} &&&
\includegraphics[width=0.07\linewidth]{2076sup-depth12-head0.png} &
\includegraphics[width=0.07\linewidth]{2076sup-depth12-head1.png} &
\includegraphics[width=0.07\linewidth]{2076sup-depth12-head2.png}
%\\
&&&
\includegraphics[width=0.07\linewidth]{1508.png} &&&
\includegraphics[width=0.07\linewidth]{1508dino-depth12-head1.png} &
\includegraphics[width=0.07\linewidth]{1508dino-depth12-head4.png} &
\includegraphics[width=0.07\linewidth]{1508dino-depth12-head5.png} &&&
\includegraphics[width=0.07\linewidth]{1508sup-depth12-head0.png} &
\includegraphics[width=0.07\linewidth]{1508sup-depth12-head1.png} &
\includegraphics[width=0.07\linewidth]{1508sup-depth12-head2.png}
\\
\includegraphics[width=0.07\linewidth]{3791.png} &&&
\includegraphics[width=0.07\linewidth]{3791dino-depth12-head0.png} &
\includegraphics[width=0.07\linewidth]{3791dino-depth12-head3.png} &
\includegraphics[width=0.07\linewidth]{3791dino-depth12-head2.png} &&&
\includegraphics[width=0.07\linewidth]{3791sup-depth12-head0.png} &
\includegraphics[width=0.07\linewidth]{3791sup-depth12-head1.png} &
\includegraphics[width=0.07\linewidth]{3791sup-depth12-head2.png}
%\\
&&&
\includegraphics[width=0.07\linewidth]{644.png} &&&
\includegraphics[width=0.07\linewidth]{644dino-depth12-head4.png} &
\includegraphics[width=0.07\linewidth]{644dino-depth12-head1.png} &
\includegraphics[width=0.07\linewidth]{644dino-depth12-head0.png} &&&
\includegraphics[width=0.07\linewidth]{644sup-depth12-head0.png} &
\includegraphics[width=0.07\linewidth]{644sup-depth12-head1.png} &
\includegraphics[width=0.07\linewidth]{644sup-depth12-head2.png}
\\
\includegraphics[width=0.07\linewidth]{2221.png} &&&
\includegraphics[width=0.07\linewidth]{2221dino-depth12-head5.png} &
\includegraphics[width=0.07\linewidth]{2221dino-depth12-head2.png} &
\includegraphics[width=0.07\linewidth]{2221dino-depth12-head0.png} &&&
\includegraphics[width=0.07\linewidth]{2221sup-depth12-head0.png} &
\includegraphics[width=0.07\linewidth]{2221sup-depth12-head1.png} &
\includegraphics[width=0.07\linewidth]{2221sup-depth12-head2.png}
%\\
&&&
\includegraphics[width=0.07\linewidth]{1323.png} &&&
\includegraphics[width=0.07\linewidth]{1323dino-depth12-head0.png} &
\includegraphics[width=0.07\linewidth]{1323dino-depth12-head5.png} &
\includegraphics[width=0.07\linewidth]{1323dino-depth12-head2.png} &&&
\includegraphics[width=0.07\linewidth]{1323sup-depth12-head0.png} &
\includegraphics[width=0.07\linewidth]{1323sup-depth12-head1.png} &
\includegraphics[width=0.07\linewidth]{1323sup-depth12-head2.png}
\\
\includegraphics[width=0.07\linewidth]{1178.png} &&&
\includegraphics[width=0.07\linewidth]{1178dino-depth12-head2.png} &
\includegraphics[width=0.07\linewidth]{1178dino-depth12-head4.png} &
\includegraphics[width=0.07\linewidth]{1178dino-depth12-head3.png} &&&
\includegraphics[width=0.07\linewidth]{1178sup-depth12-head0.png} &
\includegraphics[width=0.07\linewidth]{1178sup-depth12-head1.png} &
\includegraphics[width=0.07\linewidth]{1178sup-depth12-head2.png}
%\\
&&&
\includegraphics[width=0.07\linewidth]{578.png} &&&
\includegraphics[width=0.07\linewidth]{578dino-depth12-head5.png} &
\includegraphics[width=0.07\linewidth]{578dino-depth12-head3.png} &
\includegraphics[width=0.07\linewidth]{578dino-depth12-head1.png} &&&
\includegraphics[width=0.07\linewidth]{578sup-depth12-head0.png} &
\includegraphics[width=0.07\linewidth]{578sup-depth12-head1.png} &
\includegraphics[width=0.07\linewidth]{578sup-depth12-head2.png}
\\
\includegraphics[width=0.07\linewidth]{2281.png} &&&
\includegraphics[width=0.07\linewidth]{2281dino-depth12-head5.png} &
\includegraphics[width=0.07\linewidth]{2281dino-depth12-head4.png} &
\includegraphics[width=0.07\linewidth]{2281dino-depth12-head0.png} &&&
\includegraphics[width=0.07\linewidth]{2281sup-depth12-head0.png} &
\includegraphics[width=0.07\linewidth]{2281sup-depth12-head1.png} &
\includegraphics[width=0.07\linewidth]{2281sup-depth12-head2.png}
%\\
&&&
\includegraphics[width=0.07\linewidth]{2313.png} &&&
\includegraphics[width=0.07\linewidth]{2313dino-depth12-head2.png} &
\includegraphics[width=0.07\linewidth]{2313dino-depth12-head4.png} &
\includegraphics[width=0.07\linewidth]{2313dino-depth12-head0.png} &&&
\includegraphics[width=0.07\linewidth]{2313sup-depth12-head0.png} &
\includegraphics[width=0.07\linewidth]{2313sup-depth12-head1.png} &
\includegraphics[width=0.07\linewidth]{2313sup-depth12-head2.png}
\\
\includegraphics[width=0.07\linewidth]{3295.png} &&&
\includegraphics[width=0.07\linewidth]{3295dino-depth12-head2.png} &
\includegraphics[width=0.07\linewidth]{3295dino-depth12-head4.png} &
\includegraphics[width=0.07\linewidth]{3295dino-depth12-head5.png} &&&
\includegraphics[width=0.07\linewidth]{3295sup-depth12-head0.png} &
\includegraphics[width=0.07\linewidth]{3295sup-depth12-head1.png} &
\includegraphics[width=0.07\linewidth]{3295sup-depth12-head2.png}
%\\
&&&
\includegraphics[width=0.07\linewidth]{3377.png} &&&
\includegraphics[width=0.07\linewidth]{3377dino-depth12-head1.png} &
\includegraphics[width=0.07\linewidth]{3377dino-depth12-head0.png} &
\includegraphics[width=0.07\linewidth]{3377dino-depth12-head4.png} &&&
\includegraphics[width=0.07\linewidth]{3377sup-depth12-head0.png} &
\includegraphics[width=0.07\linewidth]{3377sup-depth12-head1.png} &
\includegraphics[width=0.07\linewidth]{3377sup-depth12-head2.png}
\\
\includegraphics[width=0.07\linewidth]{4229.png} &&&
\includegraphics[width=0.07\linewidth]{4229dino-depth12-head2.png} &
\includegraphics[width=0.07\linewidth]{4229dino-depth12-head4.png} &
\includegraphics[width=0.07\linewidth]{4229dino-depth12-head0.png} &&&
\includegraphics[width=0.07\linewidth]{4229sup-depth12-head0.png} &
\includegraphics[width=0.07\linewidth]{4229sup-depth12-head1.png} &
\includegraphics[width=0.07\linewidth]{4229sup-depth12-head2.png}
%\\
&&&
\includegraphics[width=0.07\linewidth]{581.png} &&&
\includegraphics[width=0.07\linewidth]{581dino-depth12-head5.png} &
\includegraphics[width=0.07\linewidth]{581dino-depth12-head1.png} &
\includegraphics[width=0.07\linewidth]{581dino-depth12-head3.png} &&&
\includegraphics[width=0.07\linewidth]{581sup-depth12-head0.png} &
\includegraphics[width=0.07\linewidth]{581sup-depth12-head1.png} &
\includegraphics[width=0.07\linewidth]{581sup-depth12-head2.png}
\\
\includegraphics[width=0.07\linewidth]{3059.png} &&&
\includegraphics[width=0.07\linewidth]{3059dino-depth12-head2.png} &
\includegraphics[width=0.07\linewidth]{3059dino-depth12-head4.png} &
\includegraphics[width=0.07\linewidth]{3059dino-depth12-head5.png} &&&
\includegraphics[width=0.07\linewidth]{3059sup-depth12-head0.png} &
\includegraphics[width=0.07\linewidth]{3059sup-depth12-head1.png} &
\includegraphics[width=0.07\linewidth]{3059sup-depth12-head2.png}
%\\
&&&
\includegraphics[width=0.07\linewidth]{4462.png} &&&
\includegraphics[width=0.07\linewidth]{4462dino-depth12-head2.png} &
\includegraphics[width=0.07\linewidth]{4462dino-depth12-head5.png} &
\includegraphics[width=0.07\linewidth]{4462dino-depth12-head3.png} &&&
\includegraphics[width=0.07\linewidth]{4462sup-depth12-head0.png} &
\includegraphics[width=0.07\linewidth]{4462sup-depth12-head1.png} &
\includegraphics[width=0.07\linewidth]{4462sup-depth12-head2.png}
\\
\includegraphics[width=0.07\linewidth]{1825.png} &&&
\includegraphics[width=0.07\linewidth]{1825dino-depth12-head2.png} &
\includegraphics[width=0.07\linewidth]{1825dino-depth12-head4.png} &
\includegraphics[width=0.07\linewidth]{1825dino-depth12-head5.png} &&&
\includegraphics[width=0.07\linewidth]{1825sup-depth12-head0.png} &
\includegraphics[width=0.07\linewidth]{1825sup-depth12-head1.png} &
\includegraphics[width=0.07\linewidth]{1825sup-depth12-head2.png}
%\\
&&&
\includegraphics[width=0.07\linewidth]{1971.png} &&&
\includegraphics[width=0.07\linewidth]{1971dino-depth12-head2.png} &
\includegraphics[width=0.07\linewidth]{1971dino-depth12-head4.png} &
\includegraphics[width=0.07\linewidth]{1971dino-depth12-head1.png} &&&
\includegraphics[width=0.07\linewidth]{1971sup-depth12-head0.png} &
\includegraphics[width=0.07\linewidth]{1971sup-depth12-head1.png} &
\includegraphics[width=0.07\linewidth]{1971sup-depth12-head2.png}
\\
\includegraphics[width=0.07\linewidth]{3224.png} &&&
\includegraphics[width=0.07\linewidth]{3224dino-depth12-head2.png} &
\includegraphics[width=0.07\linewidth]{3224dino-depth12-head1.png} &
\includegraphics[width=0.07\linewidth]{3224dino-depth12-head5.png} &&&
\includegraphics[width=0.07\linewidth]{3224sup-depth12-head0.png} &
\includegraphics[width=0.07\linewidth]{3224sup-depth12-head1.png} &
\includegraphics[width=0.07\linewidth]{3224sup-depth12-head2.png}
&&&
\includegraphics[width=0.07\linewidth]{393.png} &&&
\includegraphics[width=0.07\linewidth]{393dino-depth12-head5.png} &
\includegraphics[width=0.07\linewidth]{393dino-depth12-head2.png} &
\includegraphics[width=0.07\linewidth]{393dino-depth12-head3.png} &&&
\includegraphics[width=0.07\linewidth]{393sup-depth12-head0.png} &
\includegraphics[width=0.07\linewidth]{393sup-depth12-head1.png} &
\includegraphics[width=0.07\linewidth]{393sup-depth12-head2.png}
\\
\includegraphics[width=0.07\linewidth]{1525.png} &&&
\includegraphics[width=0.07\linewidth]{1525dino-depth12-head5.png} &
\includegraphics[width=0.07\linewidth]{1525dino-depth12-head0.png} &
\includegraphics[width=0.07\linewidth]{1525dino-depth12-head1.png} &&&
\includegraphics[width=0.07\linewidth]{1525sup-depth12-head0.png} &
\includegraphics[width=0.07\linewidth]{1525sup-depth12-head1.png} &
\includegraphics[width=0.07\linewidth]{1525sup-depth12-head2.png}
&&&
\includegraphics[width=0.07\linewidth]{4859.png} &&&
\includegraphics[width=0.07\linewidth]{4859dino-depth12-head3.png} &
\includegraphics[width=0.07\linewidth]{4859dino-depth12-head5.png} &
\includegraphics[width=0.07\linewidth]{4859dino-depth12-head0.png} &&&
\includegraphics[width=0.07\linewidth]{4859sup-depth12-head0.png} &
\includegraphics[width=0.07\linewidth]{4859sup-depth12-head1.png} &
\includegraphics[width=0.07\linewidth]{4859sup-depth12-head2.png}
\end{tabular}
	\caption{\textbf{Self-attention heads from the last layer.} We look at the attention map when using the \texttt{[CLS]} token as a query for the different heads in the last layer. Note that the \texttt{[CLS]} token is not attached to any label or supervision.}
\label{fig:all}
\end{figure*}

\begin{figure*}[t]
\centering
\includegraphics[width=\linewidth]{tsne-classes.pdf}
  \caption{
    t-SNE visualization of ImageNet classes as represented using \OURS.
    For each class, we obtain the embedding by taking the average feature for all images of that class in the validation set.
  }
  \label{fig:tsne}
\end{figure*}


\end{document}
